\chapter*{Lexique}
\addcontentsline{toc}{chapter}{Lexique}
Afin de faciliter la lecture, je met à disposition une liste des abréviations utilisées dans ce rapport.

\begin{longtable}{ll}
AAA & Authentication-Authorization-Accounting\\
AD & Active Directory\\
AES & Advanced Encryption Standard\\
AH & Authentication Header\\
ASA & Adaptive Security Appliance\\
ATM & Asynchronous Transfer Mode\\
CA & Certifcate Authority\\
CBC & Cipher Block Chaining\\
CHAP & Challenge-handshake Authentication Protocol\\
CRM & Customer Relationship Manager\\
DES & Data Encryption Standard\\
DFS & Distributed File System\\
DLCI & Data Link Connection Identifier\\
DMZ & Demilitarized Zone\\
ECDHE & Elliptic Curve Diffie-Hellman Exchange\\
ESP & Encapsulating Security Payload\\
ETTD & Equipement terminal de transmission de données\\
GPO & Group Policy Object\\
HDR & Header\\
IETF & Internet Engineering Task Force\\
IKE & Internet Key Exchange\\
IP & Internet Protocol\\
IPSec & Internet Protocol Security\\
IVC & Integrity Check Value\\
KE & Key Exchange\\
MAC & Message Authentication Code\\
OU & Organizational Unit\\
OWA & Outlook Web App\\
PAP & Password Authentication Protocol\\
PPP & Pont-to-Point Protocol\\
PVC & Permanent Virtual Circuit\\
SA & Security Association\\
SHA & Secure Hash Algorithm\\
SSL & Secure Sockets Layer\\
SVC & Switched Virtual Circuit\\
TLS & Transport Layer Security\\
UDP & User Datagram Protocol\\
VPN & Virtual Private Network\\
WAN & World Area Network\\
\end{longtable}