\chapter*{Conclusion}
\addcontentsline{toc}{chapter}{Conclusion}
Ce rapport abouti à la réalisation d'un comparatif des solutions d'accès distant.
Ce comparatif met en évidence les différences de philosophies de la part des fabricants.
En effet, chaque solution possède ses propres caractéristiques et fonctionnalités.
Même si les fonctionnalités sont similaires, les implémentations diffèrent.\\

Mes recommandations sont les suivantes.

La solution Juniper est de loin, la plus agréable à utiliser.
Elle offre une interface claire, et la configuration est aisée même pour un débutant.

Le guide d'administration est complet et clair.
Par contre, c'est la solution la plus chère.
À éviter pour des petits budgets.\\

La solution Windows est sans doute la plus simple à configurer, mais son fonctionnement peut être une source de vulnérabilité.
En effet, la procédure de connexion étant automatique, l'utilisateur n'a pas de possibilité d'empêcher l'établissement de la connexion entre son ordinateur et le réseau de l'entreprise.
Néanmoins, il a la possibilité de couper la connexion une fois qu'elle est établie.
Le configuration est simple du fait qu'elle est automatisée.

Par contre, le guide en ligne fourni par Microsoft n'est pas clair.
Il détaille plusieurs types d'installation, il n'est donc pas évident de trouver celle qui s'applique à son propre cas.\\

Le solution Cisco est celle à éviter.
L'interface n'est pas intuitive.
Pour qu'un changement soit effectif, il est nécessaire de passer par plusieurs fenêtres.
Des groupes par défaut viennent polluer l'interface, car ils ne sont pas supprimables.
Les groupes qui sont créés par l'administrateur héritent de toutes les propriétés du groupe par défaut.
Il est donc nécessaire soit d'activer le groupe par défaut, soit de modifier tous les champs du nouveau groupe, ce qui alourdi la tâche de l'administrateur.

Les problèmes de version sont aussi un désavantage dans ce cas.
Car ce qui marche dans une version ne marche plus dans une autre.

De plus, le guide d'administration n'est pas clair.
Il est complet, mais les informations fournies ne sont pas assez approfondies pour comprendre l'ensemble des réglages.
 