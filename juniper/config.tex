La solution de Juniper se base sur une passerelle SSL.
Dans ce cas-ci, on a utilisé une SA 2500 tournant sous JunOS 8.1.

La configuration est facile, si on suit le guide d'administration.

Commençons par un peu de terminologie.
Sur la passerelle,j'ai configuré deux rôles et un \textit{realm}.
Le \textit{realm} sert à l'authentification des utilisateurs.
Les rôles décrivent les ressources accessibles aux utilisateurs.


Ainsi dans le realm labotfe de la passerelle, les méthodes d'authentification sont sélectionnées.
J'ai choisi une authentification via le Domain Controler du domaine \textit{labotfe.be}.