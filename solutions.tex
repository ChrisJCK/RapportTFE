Le choix des solutions testées s'est fait sur base des leaders commerciaux.
Cisco et Juniper sont des fabricants de matériel réseau (switch, routeur, firewall, ...).
Microsoft est le leader commercial dans les systèmes d'exploitation.

L'utilisation d'un environnement Windows était inévitable, car les systèmes d'exploitation Windows occupent presque 80\% des parts de marché.
De plus, mon expérience est plus grande dans l'environnement Windows, surtout pour la partie Active Directory.
Il existe des solutions similaires chez Unix et Apple, mais par manque de connaissances de ces produits, il m'a semblé plus évident de construire un environnement Windows dans le cadre de ce TFE.
La solution DirectAccess est assez récente, c'était une bonne opportunité de la tester et de la découvrir.

En plus de fournir du matériel hardware, Cisco et Juniper possèdent des systèmes d'exploitation complet pour la gestion des firewalls et des passerelles.
Chaque fabricant possède sa propre philosophie.
Ainsi Juniper propose une passerelle dédiée au accès distant.
Par contre Cisco fonctionne avec du tout-en-un, la passerelle est intégrée au firewall.