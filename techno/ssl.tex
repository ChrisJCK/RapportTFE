\section{Secure sockets layer - SSL}
Netscape a lancé SSL 1 en 1994 dans le but de sécuriser des transactions réalisées avec leur navigateur. 
Dans la même année, SSL 2 était déjà en route. 
Mais le protocole montrait déjà des vulnérabilités.
Fin 1995, SSL 3 était lancé. 
Il s'agissait d'une version complètement récrite de SSL, qui introduisait de nouvelles fonctionnalités issues de PCT\footnote{Microsoft's \textit{Private Communications Technology}}.
Bien que les machines actuelles intègrent SSL 3, elles essaient d'abord de négocier une connexion en SSL 2.

SSL utilise des suites de chiffrement. 
Ces suites se composent de trois fonctions de chiffrement : la méthode d'échange de clé, l'algorithme de chiffrement et une méthode de hachage. 
Il existe un large ensemble de suite, les clients envoient la liste des suites qu'ils savent gérer au serveur. 
Ce dernier choisit, dans la liste, une suite qu'il implémente. 

OpenSSL est l'implémentation la plus courante de SSL. 
Cette implémentation possède un interface en ligne de commande, qui permet de générer des clés RSA, signer des certificats, calculer des valeurs de hash, etc..

Dans un effort de standardisation de SSL, l'IETF a lancé le protocole TLS\footnote{Transport Layer Security}.
Il se base principalement sur SSL 3 bien qu'il ne soit pas compatible avec ce dernier.
L'utilisation de SSL est déconseillé suite à de nombreuses vulnérabilités détectées ces derniers temps (FREAK (2015), Hearthbleed (2014), BEAST (2013), etc.).

\subsection{Le protocole SSL}
SSL est un protocole de sécurité de la couche Application. 
Pour éviter des erreurs lors de la transmission des données, SSL utilise le protocole TCP.

SSL est composé de quatre sous-protocoles : "Record Layer", "ChangeCipherSpec Protocol", "Alert Protocol" et "Handshake Protocol".

\subsubsection{Record Layer}
Ce protocole sert à formater les messages des autres protocoles.
Il fournit à tous les messages un en-tête et un hash.

\subsubsection{ChangeCipherSpec Protocol}
Il s'agit d'un message signalant le début d'une connexion sécurisée entre le client et le serveur.

\subsubsection{Alert Protocol}
Il émet les anomalies de la connexion entre les deux machines.
Les messages sont composés de deux champs : le niveau de gravité et la description de l'alerte

\subsubsection{Handshake Protocol}

De manière analogue à TCP, une session SSL se divise en trois phase (voir Fig.\ref{fig:ssl} p.\pageref{fig:ssl}) : 
\begin{enumerate}
	\item \'Etablissement de la connexion
	\item Transfert des données
	\item Clôture de la connexion.
\end{enumerate}
\begin{figure}[ht]
\centering
\begin{tikzpicture}
	\draw[-,ultra thick] (0,10.8) node [above] {Client} -- (0,0) ;
	\draw[-,ultra thick] (6,10.8) node [above] {Serveur}-- (6,0);
	\draw[arrow] (0,10) -- (6,10) node [midway,above] {Handshake : ClientHello};
	\draw[arrow] (6,9) -- (0,9) node [midway,above] {Handshake : ServerHello};
	\draw[arrow] (6,8.4) -- (0,8.4) node [midway,above] {Handshake : ServerKeyExchange};
	\draw[arrow] (6,7.8) -- (0,7.8) node [midway,above] {Handshake : ServerHelloDone};
	\draw[arrow] (0,6.8) -- (6,6.8) node [midway,above] {Handshake : ClientKeyExchange};
	\draw[arrow] (0,6.2) -- (6,6.2) node [midway,above] {ChangeCipherSpec};
	\draw[arrow] (0,5.6) -- (6,5.6) node [midway,above] {Handshake : Finished};
	\draw[arrow] (6,4.6) -- (0,4.6) node [midway,above] {ChangeCipherSpec};
	\draw[arrow] (6,4) -- (0,4) node [midway,above] {Handshake : Finished};
	\draw[|-|,ultra thick] (6.5,10.8) -- (6.5,3.8) node [midway,right] {\'Etablissement de la connexion};
	\draw[arrow] (0,2.6) -- (6,2.6);
	\draw[arrow] (6,2) -- (0,2) node [midway,above] {Application Data};
	\draw[|-|,ultra thick] (6.5,2.8) -- (6.5,1.8) node [midway,right] {Transfert des données};
	\draw[arrow] (0,0.8) -- (6,0.8);
	\draw[arrow] (6,0.2) -- (0,0.2) node [midway,above] {Alert Close Notify};
	\draw[|-|,ultra thick] (6.5,1) -- (6.5,0) node [midway,right] {Clôture de la session};
\end{tikzpicture}
\caption{Session SSL}
\label{fig:ssl}
\end{figure}
La session commence par le \textit{triple handshake}. 
Le client envoie un message ClientHello, qui indique la version de SSL supportée, la liste des suites de chiffrement et les algorithmes de compression. 
La version de SSL est signalée par deux champs dans l'en-tête : version mineur et version majeure. 
SSL3 a une version majeure de 3 et une version mineure de 0, et TLS a un version majeure de 3 et une version mineure de 1. 

Le serveur répond par trois messages.
\begin{enumerate}
	\item Le message \textit{ServerHello} indique au client la suite de chiffrement et l'algorithme de compression à utiliser. Il complète aussi le champ "Session ID". 
	\item Le certificat du serveur permet au client de vérifier l'identité du serveur, la validité du certificat et l'émetteur du certificat. Il contient la clé publique du serveur. Cette clé va servir à générer les différents clés pour la session.
	\item Le message \textit{ServerHelloDone} précise la fin de la séquence \textit{Hello}.
\end{enumerate}
Suite à ces trois messages, le client donne au serveur des inputs pour la génération des clés (\textit{ClientKeyExchange}).
Les messages \textit{ChangeCipherSpec} signalent que la connexion est passée de l'état non-sécurisé à l'état sécurisé. 
Les messages \textit{Finished} préviennent de la fin du handshake.
 
Le client et le serveur sont capables de s'échanger des données de manière sécurisée. 

Finalement, la session est clôturée.