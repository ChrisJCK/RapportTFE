\section{Secure sockets layer - SSL}
Netscape a lancé SSL 1 en 1994, dans le but de sécuriser des transactions réalisées avec leur navigateur. 
Dans la même année, SSL 2 était déjà en route. 
Mais le protocole montrait déjà des problèmes de sécurité.
Fin 1995, SSL 3 était lancé. 
Il s'agissait d'une version complètement récrite de SSL, qui introduisait de nouvelles fonctionnalités issues de PCT\footnote{Microsoft's \textit{Private Communications Technology}}.
Bien que les machines actuelles intègrent SSL 3, elles essaient d'abord de négocier une connexion en SSL 2.

Dans un effort de standardisation de SSL, l'IETF a lancé le protocole TLS\footnote{Transport Layer Security}.
Il se base principalement sur SSL 3 bien qu'il ne soit pas compatible avec ce dernier.

SSL utilise des suites de chiffrement. Ces suites se composent de trois fonctions de chiffrement : la méthode d'échange de clé, l'algorithme de chiffrement et une méthode de hachage. 
Il existe un large ensemble de suite, les client ont donc un mécanisme pour signaler les suites qu'ils gèrent et qu'ils utilisent. 

OpenSSL est l'implémentation la plus courante de SSL. 
Cette implémentation possède un interface en ligne de commande, qui permet de générer des clés RSA, signer des certificats, calculer des valeurs de hash, ...

\subsection{Le protocole SSL}
SSL est un protocole de la couche transport, il utilise donc les protocoles de cette couche pour le transfert des données. 
Pour éviter des problèmes lors de la transmission des données, SSL utilise le protocole TCP.

De manière analogue à TCP, une session SSL se divise en trois phase (voir Fig.\ref{fig:ssl} p.\pageref{fig:ssl}) : 
\begin{enumerate}
	\item \'Etablissement de la connexion
	\item Transfert des données
	\item Clôture de la connexion.
\end{enumerate}
\begin{figure}[ht]
\centering
\begin{tikzpicture}
	\draw[-,ultra thick] (0,10.8) node [above] {Client} -- (0,0) ;
	\draw[-,ultra thick] (6,10.8) node [above] {Serveur}-- (6,0);
	\draw[arrow] (0,10) -- (6,10) node [midway,above] {Handshake : ClientHello};
	\draw[arrow] (6,9) -- (0,9) node [midway,above] {Handshake : ServerHello};
	\draw[arrow] (6,8.4) -- (0,8.4) node [midway,above] {Handshake : Certificate};
	\draw[arrow] (6,7.8) -- (0,7.8) node [sloped, midway,above] {Handshake : ServerHelloDone};
	\draw[arrow] (0,6.8) -- (6,6.8) node [midway,above] {Handshake : KeyExchange};
	\draw[arrow] (0,6.2) -- (6,6.2) node [midway,above] {ChangeCipherSpec};
	\draw[arrow] (0,5.6) -- (6,5.6) node [midway,above] {Handshake : Finished};
	\draw[arrow] (6,4.6) -- (0,4.6) node [midway,above] {ChangeCipherSpec};
	\draw[arrow] (6,4) -- (0,4) node [midway,above] {Handshake : Finished};
	\draw[arrow] (0,2.6) -- (6,2.6);
	\draw[arrow] (6,2) -- (0,2) node [midway,above] {Application Data};
	\draw[arrow] (0,0.8) -- (6,0.8);
	\draw[arrow] (6,0.2) -- (0,0.2) node [midway,above] {Alert Close Notify};
\end{tikzpicture}
\caption{Session SSL}
\label{fig:ssl}
\end{figure}
La session commence par le \textit{triple handshake}. 
Le client envoie un message ClientHello, qui indique la version de SSL supportée, la liste des suites de chiffrement et les algorithmes de compression. 
La version de SSL est signalée par deux champs dans l'en-tête : version mineur et version majeure. 
SSL3 a une version majeure de 3 et une version mineure de 0, et TLS a un version majeure de 3 et une version mineure de 1. 

Le serveur répond par trois messages.
\begin{enumerate}
	\item Le message \textit{ServerHello} indique au client la suite de chiffrement et l'algorithme de compression à utiliser.
	\item Le certificat du serveur permet au client de vérifier l'identité du serveur et contient la clé publique du serveur. Cette clé va servir à générer les différents clés pour la session.
	\item Le message \textit{ServerHelloDone} précise la fin de la séquence \textit{Hello}.
\end{enumerate}
Suite à ces trois messages, le client donne au serveur des inputs pour la génération des clés (\textit{ClientKeyExchange}), signal au serveur qu'il utilise les nouvelles clés pour le chiffrement et l'authentification (\textit{ChangeCipherSpec}) et qu'il a fini le handshake (\textit{Finished}).
 Le serveur répond avec son message \textit{ChangeCipherSpec} et son \textit{Finished}.
 
 Le client et le serveur sont capables de s'échanger des données de façon sécurisé. 
 
 Finalement, la session est clôturée.