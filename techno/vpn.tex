\section{Virtual Private Network - VPN}
Le terme VPN est un acronyme pour « Virtual Private Network ».
Un VPN est par définition un réseau virtuel qui transfère des données privées en créant un tunnel à travers un réseau public.

Historiquement, les VPN étaient construits sur des lignes louées, mais le coût de ces infrastructures était trop important.
Maintenant, les VPN sont basés sur l'Internet dont l'avantage principal est son faible coût.
Mais en utilisant l'Internet, les données privées deviennent accessibles à tout le monde. 
Il a donc été nécessaire de fournir des protocoles permettant d'assurer la confidentialité et l'intégrité des données à travers le réseau public. 

Un tunnel VPN est monté soit entre deux passerelles VPN soit entre deux hôtes soit entre une passerelle VPN et un hôte. 
Les deux équipements sont d'un point de vue logique connectées directement l'une à l'autre. 
Le tunnel permet d'envoyer des données du réseau privé à travers le réseau public.
Il encapsule les données dans un protocole compris par les deux extrémités. 
L'émetteur encapsule les données et le destinataire récupère les données. 
En plus de l'encapsulation des données, les tunnels VPN réalisent du chiffrement. 

\subsection{Les types de VPN}
Il existe deux grands types de VPN : les VPN site-à-site et les VPN client-à-site ou VPN "remote access".

\subsubsection{Les VPN site-à-site} 
Ils sont utilisés pour connecter des sites entre eux. 
Le tunnel est monté entre deux passerelles VPN dont les configurations sont connues. 
Le réseau Internet opère comme une liaison WAN entre les sites. 
Les employées peuvent échanger des informations entre les différents sites comme s'ils sont connectés sur le site distant. 

\subsubsection{Les VPN client-à-site} 
Ils sont généralement utilisés par des travailleurs pour accéder aux ressources de l'entreprise depuis des emplacements non fiables.  
L'utilisateur se connecte via son ordinateur ou son smartphone à la passerelle VPN de son entreprise.  
Dans ce cas, la configuration n'est pas connue, car selon la localisation de l'utilisateur, les paramètres de connexion changent.  
Il est souvent nécessaire d'installer sur l'appareil mobile un client VPN. 

Ces types de VPN sont généralement des tunnels SSL/TLS en raison de leur facilité de configuration et de la mobilité qu'ils offrent.

\subsection{Les protocoles de tunneling de couche 2}
Avant de créer le tunnel, ces protocoles se mettent d'accord sur des paramètres de session, comme le protocole de chiffrement.
Une fois les paramètres acceptés par les deux extrémités, le tunnel est monté.
Les données sont envoyées via ce tunnel entre les deux extrémités.
Il est possible de transmettre des données issues de protocole non-routable à travers le réseau public. 
Les données sont encapsulés dans un autre protocole qui est routable.

Par exemple, le protocole AppleTalk n'est pas supporté par l'Internet, mais grâce au tunnel, nous encapsulons le paquet AppleTalk dans un paquet IP. 
Ce paquet est envoyé sur l'Internet qui va le transmettre au destinataire.
Ce dernier désencapsule le paquet IP et récupère le paquet AppleTalk. 

Il est nécessaire que les deux extrémités du tunnel comprennent le protocole encapsulé.
Sinon, la destination ne saura pas capable de traiter le paquet encapsulé. 

\subsubsection{Point-to-Point Tunneling Protocol - PPTP}
C'est un protocole propriétaire de Microsoft.
Il est implémenté depuis Windows 2000 dans toutes les machines Windows et il existe des clients PPTP pour Linux et OS X.
Sa configuration est simple, mais la sécurité qu'il propose n'est pas exempt de tout reproche.

Ce protocole encapsule des frames PPP\footnote{Point-to-Point Protocol} dans un tunnel IP.
Il fonctionne en quatre phases dont une est optionnelle : Link Establishment Phase, Authentication Phase, Callback Control Phase, Network Control Phase.
\begin{enumerate}
	\item La phase une sert à établir, maintenir et terminer la connexion physique entre les deux hôtes. C'est aussi à ce moment que les protocoles d'authentification sont choisis.
	\item Le client PPTP est authentifié en utilisant le protocole PPP. L'authentification peut se faire en claire ou en utilisant des protocoles comme PAP et CHAP. 
	\item La phase trois est optionnel et elle permet une sécurité accrue. Elle déconnecte le client et le serveur. Ensuite, le serveur rappelle le client. 
	\item La dernière phase sert à négocier et implémenter les protocoles de compression et de chiffrement.
\end{enumerate}
\subsubsection{Layer 2 Tunneling Protocol - L2TP}
Ce protocole a été normalisé en utilisant les avantages des protocoles PPTP et L2F de Cisco\footnote{Layer 2 Forwarding}. 
Mais il ne permet toujours pas la confidentialité du trafic. 
Il est possible de faire de l'authentification et du chiffrement avec les paquets PPP, mais la connexion reste vulnérable au niveau de la couche transport. 
Il est donc intéressant d'associer L2TP avec un autre protocole de sécurité comme IPSec.

PPTP  et L2TP ne possèdent pas un niveau de sécurité élevé pour être utilisé comme tunnel VPN à travers Internet.
Cependant, PPTP est utilisé car il est simple à configurer et disponible sur la plupart des plateformes.
Ils sont le plus souvent remplacer par des tunnels IPSec ou SSL, qui possèdent un meilleur niveau de sécurité.