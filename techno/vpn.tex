\section{Virtual Private Network - VPN}
Le terme VPN est un acronyme pour « Virtual Private Network ».
Un VPN est par définition un réseau virtuel qui transfère des données privées en créant un tunnel à travers un réseau public.

Historiquement, les VPN étaient construits sur des lignes louées, mais le coût de ces infrastructures était trop important.
Maintenant, les VPN sont construits sur l'Internet.
L'avantage principal est son faible coût.
Mais en utilisant l'Internet, les données sont accessibles à tout le monde. 
Il a donc été nécessaire de fournir des protocoles permettant d'assurer la confidentialité et l'intégrité des données à travers le réseau public. 

Un tunnel VPN est monté entre deux passerelles VPN, ces passerelles sont d'un point de vue logique connectées directement l'une à l'autre comme illustré sur la figure. 
Le tunnel permet d'envoyer des données du réseau privé à travers le réseau public.
Le tunnel encapsule les données dans un protocole compris par les deux passerelles. 
La passerelle émettrice encapsule les données et la passerelle destinataire récupère les données. 
En plus de l'encapsulation des données, les tunnels VPN peuvent réaliser du chiffrement. 
Le chiffrement permet de rendre les données  inutilisables en cas de vol. 

\subsection{Les types de VPN}
Il existe deux grands types de VPN : les VPN site-à-site et les VPN client-à-site.  

\subsubsection{Les VPN site-à-site} 
Ils sont utilisés pour connecter des sites entre eux. 
Le tunnel est monté entre deux passerelles VPN dont les configurations sont connues. 
Le réseau Internet opère comme une liaison WAN entre les sites. 
Les employées peuvent échanger des informations entre les différents sites comme s'ils sont connectés sur le site distant. 
Dans ce cas-ci, nous utilisons des tunnels VPN IPsec. 
Je discute d'IPsec plus loin dans ce chapitre. 

\subsubsection{Les VPN client-à-site} 
Ils sont généralement utilisés par des travailleurs pour accéder aux ressources de l'entreprise depuis des emplacements non fiables.  
L'utilisateur se connecte via son ordinateur ou son smartphone à la passerelle VPN de son entreprise.  
Dans ce cas, la configuration n'est pas connue, car selon la localisation de l'utilisateur, les paramètres de connexion changent.  
Il est souvent nécessaire d'installer sur l'appareil mobile un client VPN. 
Nous utilisons des tunnels VPN SSL pour la simplicité de configuration. 
Je discute de SSL plus loin dans ce chapitre.

\subsection{Les protocoles de tunneling}
Ces protocoles permettent de transférer les paquets d'un protocole à l'intérieur d'un autre protocole. 
On parle d'encapsulation.  
Il est nécessaire que les deux extrémités du tunnel comprennent le protocole encapsulé.  
Il existe plusieurs protocoles, je ne discuterai que de ceux utilisés actuellement.
\subsubsection{Point-to-Point Tunneling Protocol (PPTP)}
PPTP est un protocole de tunneling. 
Il permet de router n'importe qu'elle protocole à travers le réseau IP. 
Il fonctionne en quatre phases dont une est optionnelle : Link Establishment Phase, Authentication Phase, Callback Control Phase, Network Control Phase.
\begin{enumerate}
	\item La phase une sert à établir, maintenir et terminer la connexion physique entre les deux hôtes. C'est aussi à ce moment que les protocoles d'authentification sont choisis.
	\item La phase une sert à établir, maintenir et terminer la connexion physique entre les deux hôtes. C'est aussi à ce moment que les protocoles d'authentification sont choisis.
	\item La phase trois est optionnel et elle permet une sécurité accrue. Elle déconnecte le client et le serveur. Ensuite, le serveur rappelle le client. 
	\item La dernière phase sert à négocier et implémenter les protocoles de compression et de chiffrement.
\end{enumerate}
\subsubsection{Layer 2 Tunneling Protocol (L2TP)}
Ce protocole a été créé en utilisant les avantages des protocoles PPTP et L2F. 
Mais il ne permet toujours pas la confidentialité du trafic. 
Il est possible de faire de l'authentification et du chiffrement avec les paquets PPP, mais la connexion reste vulnérable au niveau de la couche transport. 
Il est donc intéressant d'associer L2TP avec un autre protocole de sécurité comme IPSec.