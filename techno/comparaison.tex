Ce chapitre a pour objectif d'établir un comparatif théorique des technologies décrites précédemment. 
Ce comparatif se base sur des critères de facilité, de fonctionnalité et de sécurité.
\section{Les critères de comparaison}
Tous les protocoles décrits dans le chapitre \ref{ch:dscrp} visent un même but, mais en ayant des implémentations différentes.
Pour pouvoir les différencier, il est utile de réaliser un comparatif sur des critères pertinents.
Les critères sélectionnés sont : 
\begin{itemize}
	\item La facilité de configuration
	\item Le type de VPN
	\item L'intégrité des données
	\item L'authentification
	\item Le chiffrement
\end{itemize}
D'un point de vue purement technique, la facilité de configuration précise le degré de complexité pour mettre en place un tunnel VPN en utilisant une technologie.
Une note entre 1 et 5 est attribuée, 1 désignant que la configuration est difficile et 5 qu'elle est facile.
Le type de VPN précise si c'est un tunnel pour du Remote Access, du site à site, ou les deux.
Les trois derniers critères vérifient la présence ou non de la fonctionnalité.

\section{Tableau de comparaison}
\begin{center}
\begin{longtable}{|m{2cm}|m{2cm}|m{2cm}|m{3cm}|m{3cm}|m{2cm}|}
\toprule 
Protocoles & Facilité de configuration & Type de VPN & Intégrité des données & Authentification & Chiffrement\\
\hline
 HTTPS & 2 & Remote & \ok & \ok & \ok \\
 \hline
 IPSec (ESP, tunnel) & 4 & les deux & \ok & \ok & \ok \\
 \hline
 SSH & 5 & Remote & \ok & \ok & \ok \\
 \hline
 SSTP & 4 & Remote & \ok & \ok & \ok \\
 \hline
 PPTP/L2TP & 5 & Remote & \nok & \ok & \nok \\
 \bottomrule
 \caption{Tableau de comparaison théorique des technologies d'accès à distance}
 \label{tab:techno}\\
\end{longtable}
\end{center}
PPTP et L2TP ne fournissent pas de chiffrement, ni de vérification de données de par leur conception.
Mais ils utilisent un autre protocole pour combler ces lacunes.
Ainsi les tunnels L2TP sont souvent associés à IPSec pour garantir l'intégrité et le chiffrement des données.