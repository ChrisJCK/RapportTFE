Dans ce chapitre, je réalise un comparatif théorique des technologies décrites dans le chapitre précédent.
Je précise, dans un premier temps, les critères de comparaison.
Puis, je présente le tableau.
Finalement, je réalise une analyse de ce tableau.
\section{Les critères de comparaison}
Tous les protocoles décrits dans le chapitre \ref{ch:dscrp} visent un même but, mais en ayant des implémentations différentes.
Pour pouvoir les différencier, il est utile de réaliser un comparatif sur des critères pertinents.
Les critères que j'ai sélectionné sont : 
\begin{itemize}
	\item La facilité de configuration
	\item Le type de VPN
	\item La couche de protection
	\item L'intégrité des données
	\item L'authentification
	\item Le chiffrement
\end{itemize}
D'un point de vue purement technique, la facilité de configuration précise le degré de complexité pour mettre en place un tunnel VPN en utilisant une technologie.

\section{Tableau de comparaison}
\begin{center}
\bottomcaption{Tableau de comparaison théorique des technologies d'accès à distance}
\label{tab:techno}
\tablehead{\hline Protocoles & Facilité de configuration\\}
\begin{xtabular}{|c|c|}
\hline
 IPSec (AH, tunnel) & ++\\
 \hline
 IPSec (ESP, tunnel) & ++ \\
 \hline
 IPSec (AH, transport) & +++ \\
 \hline
 IPSec (ESP, transport) & +++ \\
 \hline
 SSL/TLS & ++++\\
 \hline
 SSH & ++++ \\
 \hline
 SSTP & ++++ \\
 \hline
 HTTPS & +++++ \\
 \hline
\end{xtabular}
\end{center}