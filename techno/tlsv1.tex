\subsubsection{Basic handshake}
\begin{enumerate}
\item Le client envoie un message \texttt{ClientHello}, qui contient les versions de TLS supportées, une liste des suites de chiffrement.

\item Le serveur répond par un message \texttt{ServerHello}. 
Ce message contient la version de TLS et la suite de chiffrement à utiliser.
Il envoie un message \texttt{Certificate} ou \texttt{ServerKeyExchange} si la suite de chiffrement l'exige.
Et il transmet un message \texttt{ServerHelloDone} pour signaler la fin des négociations.

\item Le client répond avec un message \texttt{ClientKeyExchange}. Ce message contient soit un \textit{PreMasterSecret}, soit une clé publique, soit rien. 

Le client et le serveur utilisent les éléments reçus précédemment pour générer un secret maitre. Ce secret sert à créer toutes les autres clés qui seront utilisées pour cette connexion.

\item Le client envoie un enregistrement \texttt{ChangeCipherSpec} dont le \textit{content type} est $20$. 
Cet enregistrement signale au serveur que les messages du client seront authentifiés et chiffrés à l'avenir par la suite de chiffrement définit lors du \textit{handshake}.

\item Finalement, le client émet un message \texttt{Finished}. 
Ce messages est authentifié et chiffré.
Si le serveur n'arrive pas à vérifier ce message, il ferme la connexion.

Le serveur envoie également un message \texttt{Finished}.

 \item Le client et le serveur peuvent s'envoyer les données applicatives en tout sécurité.
L'enregistrement utilisé possède un \textit{content type} de $23$.
\end{enumerate}
\subsubsection{Client-authenticated handshake}
Le déroulement est identique à celui du \textit{basic handshake}.
Sauf qu'avant que le serveur envoie son \texttt{ServerHelloDone}, il demande le certificat du client via le message \texttt{CertificateRequest}.

Le client répond en envoyant son certificat dans le message \texttt{Certificate}. 
De plus, il émet un message \texttt{CertificateVerify}.
Ce dernier est signé par le clé privée du client. 
Le serveur vérifie cette signature en utilisant la clé publique contenu dans le certificat du client reçue auparavant.
\begin{figure}[ht]
\centering
\begin{tikzpicture}
	\draw[-,draw=none] (0,10.8) node [above] {Client} -- (0,-0.6) ;
	\draw[-,draw=none] (6,10.8) node [above] {Serveur}-- (6,-0.6);
	\draw[arrow] (0,10) -- (6,10) node [midway,above] {ClientHello};
	\draw[arrow] (6,9) -- (0,9) node [midway,above] {ServerHello};
	\draw[arrow] (6,8.4) -- (0,8.4) node [midway,above] {Certificate*};
	\draw[arrow] (6,7.8) -- (0,7.8) node [midway,above] {ServerKeyExchange*};
	\draw[arrow] (6,7.2) -- (0,7.2) node [midway,above] {CertificateRequest*};
	\draw[arrow] (6,6.6) -- (0,6.6) node [midway,above] {ServerHelloDone};
	\draw[arrow] (0,5.6) -- (6,5.6) node [midway,above] {Certificate*};
	\draw[arrow] (0,5) -- (6,5) node [midway,above] {ClientKeyExchange};
	\draw[arrow] (0,4.4) -- (6,4.4) node [midway,above] {CertificateVerify*};
	\draw[arrow] (0,3.8) -- (6,3.8) node [midway,above] {[ChangeCiperSpec]};
	\draw[arrow] (0,3.2) -- (6,3.2) node [midway,above] {Finished};
	\draw[arrow] (6,2.2) -- (0,2.2) node [midway,above] {[ChangeCipherSpec]};
	\draw[arrow] (6,1.6) -- (0,1.6) node [midway,above] {Finished};
	\draw[arrow] (0,0.7) -- (6,0.7);
	\draw[arrow] (6,0.1) -- (0,0.1) node [midway,above] {Application Data};
	\node [text width=2.5cm] at (8,6) {* : Dépend de la suite de chiffrement};
	\node [text width=2.5cm] at (8,4) {[...] : hors \textit{handshake}};
\end{tikzpicture}
\caption{\'Echange pour un \textit{handshake} TLS}
\label{fig:tlshandshake}
\end{figure}

\subsubsection{L'ID de session}
Le message \texttt{ServerHello} contient un id de session. 
Cet id est lié aux paramètres de chiffrement dont le secret maitre. 
Il est surtout utiliser lors d'une reconnexion pour éviter toute la procédure du \textit{handshake}.
Cette dernière peut être lourde en terme de ressources pour le serveur. 

Un schéma est disponible voir Fig.\ref{fig:tlsSessionID} p.\pageref{fig:tlsSessionID}.

\begin{figure}[ht]
\centering
\begin{tikzpicture}
	\draw[-,draw=none] (0,6.6) node [above] {Client} -- (0,0) ;
	\draw[-,draw=none] (6,6.6) node [above] {Serveur}-- (6,0);
	\draw[arrow] (0,6) -- (6,6) node [midway,above] {ClientHello};
	\draw[arrow] (6,5) -- (0,5) node [midway,above] {ServerHello};
	\draw[arrow] (6,4.4) -- (0,4.4) node [midway,above] {[ChangeCiperSpec]};
	\draw[arrow] (6,3.8) -- (0,3.8) node [midway,above] {Finished};
	\draw[arrow] (0,2.8) -- (6,2.8) node [midway,above] {[ChangeCipherSpec]};
	\draw[arrow] (0,2.2) -- (6,2.2) node [midway,above] {Handshake : Finished};
	\draw[arrow] (0,1.2) -- (6,1.2);
	\draw[arrow] (6,0.6) -- (0,0.6) node [midway,above] {Application Data};
\end{tikzpicture}
\caption{\'Echange pour un \textit{handshake} TLS avec un ID de session}
\label{fig:tlsSessionID}
\end{figure}
\subsubsection{Le ticket de session}
Le ticket de session permet à des serveurs de remettre en état une session SSL sans que l'état de la session soit enregistré sur le serveur.
Le ticket est transmis au client du façon sécurisé. 
Ce dernier va le retransmettre au serveur lorsqu'il se reconnectera au serveur.

Un schéma de l'échange est disponible à la Fig.\ref{fig:tlsSessionTicket} p.\pageref{fig:tlsSessionTicket}.

Le client prévient le serveur qu'il supporte ce système en ajoutant dans le \texttt{ClientHello} une extension \texttt{SessionTicket SSL}.

Le serveur stocke dans un ticket les informations sur la session (suite de chiffrement et le secret maitre).
Le ticket est chiffré par une clé connue uniquement du serveur.
Il est ensuite transmis au client en utilisant le message \texttt{NewSessionTicket}. 
Le message est envoyé avant que le serveur envoie son \texttt{ChangeCipherSpec} et après que le serveur ait vérifié l'authenticité du client.

\begin{figure}[ht]
\centering
\begin{tikzpicture}
	\draw[-,draw=none] (0,8.8) node [above] {Client} -- (0,0) ;
	\draw[-,draw=none] (6,8.8) node [above] {Serveur}-- (6,0);
	\draw[arrow, draw=none] (0,7.8) -- (6,7.8) node [midway,above] {ClientHello};
	\draw[arrow] (0,7.2) -- (6,7.2) node [midway,above] {(SessionTicket extension)};
	\draw[arrow, draw=none] (6,6.2) -- (0,6.2) node [midway,above] {ServerHello};
	\draw[arrow] (6,5.6) -- (0,5.6) node [midway,above] {(empty SessionTicket extension)};
	\draw[arrow] (6,5) -- (0,5) node [midway,above] {NewSessionTicket};
	\draw[arrow] (6,4.4) -- (0,4.4) node [midway,above] {[ChangeCiperSpec]};
	\draw[arrow] (6,3.8) -- (0,3.8) node [midway,above] {Finished};
	\draw[arrow] (0,2.8) -- (6,2.8) node [midway,above] {[ChangeCipherSpec]};
	\draw[arrow] (0,2.2) -- (6,2.2) node [midway,above] {Handshake : Finished};
	\draw[arrow] (0,1.2) -- (6,1.2);
	\draw[arrow] (6,0.6) -- (0,0.6) node [midway,above] {Application Data};
\end{tikzpicture}
\caption{\'Echange pour un \textit{handshake} TLS avec un ticket de session}
\label{fig:tlsSessionTicket}
\end{figure}