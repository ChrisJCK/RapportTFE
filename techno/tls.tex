\section{Transport Layer Security - TLS}
TLS est le successeur de SSL.
À l'heure actuelle, TLS est en version 1.2\footnote{RFC 5246 - \url{http://tools.ietf.org/html/rfc5246}}. 
La version 1.3 est en \textit{draft}.
De la même manière que SSL, TLS a pour objectif de sécuriser les communications entre deux entités sur un réseau.

\subsection{Détails du protocole}
TLS se base sur des enregistrements pour l'échange de données, appelé "TLS Record".
Un enregistrement est caractérisé par un champ \textit{content type}, un champ version et un champ longueur.

Avant de pouvoir échanger des données applicatives, TLS réalise le "TLS Handshake".

\subsubsection{TLS Record Protocol}
Il sert à négocier une connexion sécurisée et privée entre le client et le serveur.
Bien qu'il peut être utilisé sans chiffrement, il utiliser des clés symétriques pour s'assurer que la connexion est privée.
La sécurisation de la connexion est garantie par l'utilisation d'une fonction de hachage.

\subsubsection{TLS Handshake Protocol}
Il permet l'authentification lors du début de la communication entre le serveur et le client.
Il utilise le même handshake que SSL, mais il fournit une authentification du serveur et, en option, du client.

\subsection{Différence entre SSL et TLS}
Les deux protocoles sont semblables, mais ne sont pas compatibles.
Il existe des différences entre eux.

TLS utilise un MAC\footnote{Message Authentication Code} standardisé.
Il n'est plus limité à MD5 ou SHA, et de part sa conception, il est capable d'utiliser n'importe quelle fonction.

Ils supportent des suites de chiffrements différentes.
SSL supporte RSA, Diffie-Hellman et Fortezza.
TLS ne supporte pas Fortezza, mais il est possible d'ajouter de nouvelle suite de chiffrement au futur version du protocole.