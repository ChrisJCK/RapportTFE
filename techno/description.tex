Dans ce chapitre, je présente les technologies d'accès distants.
Ces technologies sont bien connues, mais les implémentations des ces dernières par les différents acteurs varient de l'une à l'autre.
Pour commencer, il me semble utile de définir les termes VPN et tunnel VPN.
Par la suite, il est question d'expliquer et de détailler l'ensemble des protocoles liés à la sécurité des données.

\section{Virtual Private Network - VPN}
Le terme VPN est un acronyme pour « Virtual Private Network ».
Un VPN est par définition un réseau virtuel qui transfère des données privées en créant un tunnel à travers un réseau public.

Historiquement, les VPN étaient construits sur des lignes louées, mais le coût de ces infrastructures était trop important.
Maintenant, les VPN sont construits sur l'Internet.
L'avantage principal est son faible coût.
Mais en utilisant l'Internet, les données sont accessibles à tout le monde. 
Il a donc été nécessaire de fournir des protocoles permettant d'assurer la confidentialité et l'intégrité des données à travers le réseau public. 

Un tunnel VPN est monté entre deux passerelles VPN, ces passerelles sont d'un point de vue logique connectées directement l'une à l'autre comme illustré sur la figure. 
Le tunnel permet d'envoyer des données du réseau privé à travers le réseau public.
Le tunnel encapsule les données dans un protocole compris par les deux passerelles. 
La passerelle émettrice encapsule les données et la passerelle destinataire récupère les données. 
En plus de l'encapsulation des données, les tunnels VPN peuvent réaliser du chiffrement. 
Le chiffrement permet de rendre les données  inutilisables en cas de vol. 

\subsection{Les types de VPN}
Il existe deux grands types de VPN : les VPN site-à-site et les VPN client-à-site.  

\subsubsection{Les VPN site-à-site} 
Ils sont utilisés pour connecter des sites entre eux. 
Le tunnel est monté entre deux passerelles VPN dont les configurations sont connues. 
Le réseau Internet opère comme une liaison WAN entre les sites. 
Les employées peuvent échanger des informations entre les différents sites comme s'ils sont connectés sur le site distant. 
Dans ce cas-ci, nous utilisons des tunnels VPN IPsec. 
Je discute d'IPsec plus loin dans ce chapitre. 

\subsubsection{Les VPN client-à-site} 
Ils sont généralement utilisés par des travailleurs pour accéder aux ressources de l'entreprise depuis des emplacements non fiables.  
L'utilisateur se connecte via son ordinateur ou son smartphone à la passerelle VPN de son entreprise.  
Dans ce cas, la configuration n'est pas connue, car selon la localisation de l'utilisateur, les paramètres de connexion changent.  
Il est souvent nécessaire d'installer sur l'appareil mobile un client VPN. 
Nous utilisons des tunnels VPN SSL pour la simplicité de configuration. 
Je discute de SSL plus loin dans ce chapitre.

\subsection{Les protocoles de tunneling}
Ces protocoles permettent de transférer les paquets d'un protocole à l'intérieur d'un autre protocole. 
On parle d'encapsulation.  
Il est nécessaire que les deux extrémités du tunnel comprennent le protocole encapsulé.  
Il existe plusieurs protocoles, je ne discuterai que de ceux utilisés actuellement.
\subsubsection{Point-to-Point Tunneling Protocol (PPTP)}
PPTP est un protocole de tunneling. 
Il permet de router n'importe qu'elle protocole à travers le réseau IP. 
Il fonctionne en quatre phases dont une est optionnelle : Link Establishment Phase, Authentication Phase, Callback Control Phase, Network Control Phase.
\begin{enumerate}
	\item La phase une sert à établir, maintenir et terminer la connexion physique entre les deux hôtes. C'est aussi à ce moment que les protocoles d'authentification sont choisis.
	\item La phase une sert à établir, maintenir et terminer la connexion physique entre les deux hôtes. C'est aussi à ce moment que les protocoles d'authentification sont choisis.
	\item La phase trois est optionnel et elle permet une sécurité accrue. Elle déconnecte le client et le serveur. Ensuite, le serveur rappelle le client. 
	\item La dernière phase sert à négocier et implémenter les protocoles de compression et de chiffrement.
\end{enumerate}
\subsubsection{Layer 2 Tunneling Protocol (L2TP)}
Ce protocole a été créé en utilisant les avantages des protocoles PPTP et L2F. 
Mais il ne permet toujours pas la confidentialité du trafic. 
Il est possible de faire de l'authentification et du chiffrement avec les paquets PPP, mais la connexion reste vulnérable au niveau de la couche transport. 
Il est donc intéressant d'associer L2TP avec un autre protocole de sécurité comme IPSec.
\section{Internet Protocol Security - IPSec}
IPsec est un ensemble de protocole visant à sécuriser les données au niveau de la couche réseau. 
Il est composé de trois protocoles : AH (\textit{Authentication Header}), ESP (\textit{Encapsulating Security Payload}) et IKE (\textit{Internet Key Exchange}).
IKE est utilisé lors de la négociation des paramètres du tunnel VPN. 
Les deux autres protocoles fournissent la sécurité des données en les encapsulant au sein du tunnel VPN. 

\subsection{Les protocoles AH et ESP}
Le protocole AH permet uniquement l'authentification. 
L'authentification se fait à l'aide d'un algorithme MAC \footnote{Message Authentication Code} tel que MD5 ou SHA-1.
Cette algorithme prend en input des éléments de l'en-tête IP et calcule un IVC \footnote{Integrity Check Value}.
L'IVC est ajouté dans l'en-tête AH et ce dernier est ajouté au paquet IP.

Le protocole ESP permet l'authentification et le chiffrement des données.

Un schéma des paquets formés est disponible sur la Fig.\ref{fig:ipsecHead} p.\pageref{fig:ipsecHead}.

\begin{figure}
	\centering
	\includegraphics[width=16cm]{techno/IPSec-AH-ESP}
	\caption{Format des paquets IPSec}
	\label{fig:ipsecHead}
\end{figure}

\subsection{Les modes transport et tunnel}
Le mode transport est utilisé pour des connexions bouts-à-bouts entre deux hôtes.

Le mode tunnel est préféré lorsque l'un ou les deux hôtes de la communication sont des passerelles VPN.

\subsection{IPSec et le NAT}
Comme vu précédemment (Fig.\ref{fig:ipsecHead} p.\pageref{fig:ipsecHead}), IPSec protège les en-têtes des paquets IP.
Pour le protocole AH, le moindre changement au niveau des adresses IP provoque une erreur dans l'IVC vu que l'authentification s'applique au paquet entier.
Pour le protocole ESP, l'en-tête ESP est un paquet de la couche réseau. Il ne possède pas d'information sur les ports, qui sont des éléments de la couche transport.
Il n'est pas possible d'associer un port unique pour cette communication.

Pour palier à ces problèmes, une solution est d'encapsuler les paquets IPSec dans un autre paquet.
Ainsi le NAT-T (NAT-Traversal) encapsule les paquets IPSec dans un paquet UDP avec le port 4500 (par défaut).
Ainsi, c'est l'en-tête UDP qui sera modifier lors du transfert entre les deux hôtes et le paquet IPSec ne sera pas modifié.

\subsection{Security association}
Quand nous parlons de monter un tunnel, en réalité, nous synchronisons un état partagé entre les terminaisons du tunnel. 
Cet état partagé se nomme une SA (\textit{security association}) en IPSec. 
Une SA contient l'algorithme de chiffrement utilisé et les clés utilisées, l'algorithme d'authentification, un numéro d'identifiant, le \textit{security parameter index} (SPI), … 
Plus d'autres paramètres qui servent à maintenir les tunnels VPN. 
Les SA peuvent être créées manuellement ou gérées par l'IKE. 
Chaque terminaison possède deux SA, une pour le trafic entrant et une pour le trafic sortant. 
De plus, chaque paire est liée à un protocole. 
Les SA se caractérisent par un triplet formé du SPI, de l'adresse de destination et du protocole. 
Les SA sont stockées dans une SAD (\textit{security association database}). 
Cette SAD est utilisée pour déterminer quel protocole est utilisé pour les paquets sortants et pour fournir les paramètres pour déchiffrer et/ou authentifier les paquets entrants. 
Il est possible de combiner les SA pour créer des tunnels VPN complexe. 

Les SA sont des éléments simples, c'est-à-dire qu'elles traitent tous les paquets de la même manière. 
Pour un réglage plus fin, IPSec utilise des policies. 
Ces policies se basent sur les champs suivants des en-têtes du paquet.
\begin{itemize}
\item L'adresse de destination
\item L'adresse source
\item Le protocole de la couche transport
\item Le port source
\item Le port de destination
\end{itemize}
Elles servent à déterminer quels paquets à émettre sur quel tunnel, à dropper les paquets ne correspondant à aucune des règles décrites dans les policies. 
De la même manière que les SA, les policies sont stockées dans une SPD (\textit{security policy database}). 
Le fonctionnement est similaire, pour chaque paquet entrant ou sortant, le système consulte la SPD pour déterminer les règles à appliquer au paquet. Si une règle est trouvée, le système cherche après la SA correspondante.

\subsection{Le protocole IKE}
IKE a un seul objectif : procéder à des échanges de clé Diffie-Hellman pour sécuriser un tunnel VPN. 
Il négocie le chiffrement, l'authentification nécessaire au tunnel, qui satisfont les policies. 

IKE dérive du \textit{Internet Security Association and Key Management} Protocol (ISAKMP). 
ISAKMP est un framework qui fournit des outils pour la sécurisation des échanges et l'échange de clé. 
De plus, IKE utilise différents mode du protocole OAKLEY.
Il établit une SA en deux phases et il possède cinq modes d'échange, dont trois découlent d'ISAKMP.
Les deux derniers modes ne sont utilisés que lors de la phase deux.

\subsubsection{La phase 1 d'IKE}
La phase 1 crée un canal sécurisé entre les terminaux du tunnel pour déterminer les SA. 
Le canal sécurisé est créé après l'authentification des terminaux. 
Les SA de la phase 1 sont bidirectionnelles, c'est-à-dire qu'une SA sécurise le trafic entrant et sortant. 
Pour l'échange des SA de la phase 1, IKE possède deux modes d'échanges : 
\begin{itemize}
	\item Main mode
	\item Agressive mode
\end{itemize}

Le mode \textit{"main"} d'IKE travaille en trois étapes (voir Fig.\ref{fig:ipsmain} p.\pageref{fig:ipsmain}).
Il est utilisé lorsque les terminaux possèdent des adresses IP statiques. 
\begin{figure}[ht]
\centering
\begin{tikzpicture}
	\draw[-,ultra thick] (0,7) node [above] {Initiator} -- (0,0) ;
	\draw[-,ultra thick] (7,7) node [above] {Responder}-- (7,0);
	\draw[arrow] (0,6) -- (7,6) node [midway,above] {HDR - SA};
	\draw[arrow] (7,5) -- (0,5) node [midway,above] {HDR - SA};
	\draw[arrow] (0,4) -- (7,4) node [midway,above] {HDR - KE - NONCE$_i$};
	\draw[arrow] (7,3) -- (0,3) node [midway,above] {HDR - KE - NONCE$_r$};
	\draw[arrow] (0,2) -- (7,2) node [midway,above] {HDR - ID$_i$ - \textit{AUTH}};
	\draw[arrow] (7,1) -- (0,1) node [midway,above] {HDR - ID$_r$ - \textit{AUTH}};
\end{tikzpicture}
\caption{IPsec : mode "main"}
\label{fig:ipsmain}
\end{figure}
Premièrement, l'initiateur envoie un message contenant une liste des méthodes de sécurisation qu'il utilise. 
Le receveur choisit dans la liste reçue la méthode correspondant à ses policies et envoie sa décision à l'initiateur. 
Ensuite, ce dernier envoie sa clé privée pour créer le secret partagé de l'algorithme de Diffie-Hellman. 
Le receveur fait de même. Ils sont donc capables tous les deux de créer les clés. 
Les clés dépendent des méthodes d'authentification choisies. 
Finalement, l'initiateur envoie son identité et des informations sur l'authentification. 
Ces messages sont chiffrés et masquent donc l'identité des terminaux. 
L'échange se fait en six messages.

À la fin de ce mode, les terminaux sont d'accord sur les algorithmes de chiffrement et de confidentialité des données. 
Ils possèdent également les clés pour les algorithmes sélectionnés. 

Le mode \textit{"agressive"} fait le même travail de façon plus rapide, il n'utilise que trois messages (voir Fig.\ref{fig:ipsagg} p.\pageref{fig:ipsagg}).
Il est utilisé lorsque l'un des deux terminaux de la connexion possède une adresse IP dynamique.
\begin{figure}[ht]
\centering
\begin{tikzpicture}
	\draw[-,ultra thick] (0,4) node [above] {Initiator} -- (0,0) ;
	\draw[-,ultra thick] (8,4) node [above] {Responder}-- (8,0);
	\draw[arrow] (0,3) -- (8,3) node [midway,above] {HDR - SA - KE - NONCE$_i$ - ID$_i$};
	\draw[arrow] (8,2) -- (0,2) node [midway,above] {HDR - SA - KE - NONCE$_r$ - ID$_r$ - \textit{AUTH}};
	\draw[arrow] (0,1) -- (8,1) node [midway,above] {HDR - \textit{AUTH}};
\end{tikzpicture}
\caption{IPsec : mode "agressive"}
\label{fig:ipsagg}
\end{figure} 
Lors du premier envoi, l'initiateur émet la liste des méthodes de sécurisation, son identité et sa clé. 
Le receveur répond par son choix de sécurisation, sa clé, son identité et ses identifiants. 
Finalement l'initiateur s'authentifie auprès du receveur. 
Ce dernier message peut être chiffré.

Pour l'authentification d'un terminal, il existe trois méthodes : 
\begin{itemize}
	\item Kerberos
	\item Certificats	
	\item Public Key Infrastructure (PKI)
\end{itemize}

\paragraph{Kerberos}
Kerberos est un protocole réseau d'authentification par clé chiffré.
Il fournit une forte authentification à condition que tous les services utilisent Kerberos.
Il a été conçue par le MIT\footnote{Massachusetts Institute of Technology} à la fin des années 80.
Actuellement, il est recommandé d'utiliser la version 5, qui corrige des bugs critiques de la version 4.
Kerberos est disponible sur la plupart des systèmes d'exploitation actuels.
Une version gratuite est disponible sur le site du MIT, et il existe des versions commerciales.

Pour qu'un hôte puisse se connecter à un service, le système vérifie son identité, et le cas échéant accepte ou refuse la connexion.
Ce système repose sur deux serveurs sécurisés, un serveur d'authentification (AS\footnote{Authentication Server}) et un serveur d'octroi de ticket (TGS\footnote{Ticket-Granting Server}).
Ils forment dans la terminologie Kerberos un centre de distribution de ticket ou KDC\footnote{Key Distribution Center} en anglais.
De plus, Kerberos utilise des clés chiffrées pour toutes les communications.

\paragraph{Certificat}
Un certificat n'est rien de plus qu'une clé publique accompagnée d'un identifiant dont l'ensemble des informations sont signées par un tiers de confiance.
Ce tiers est ce que l'on appelle une autorité certificative (CA). 
Elle est reconnue au niveau mondiale.
Un utilisateur envoie sa clé publique au CA et reçoit en retour, après vérification, son certificat signé par la CA.

\paragraph{PKI}
Une PKI est un ensemble de composants et de processus informatiques, humains et techniques dont le but est de gérer la distribution et la vie des certificats.

\subsubsection{La phase 2 d'IKE}
Une fois la SA établie, les terminaux peuvent l'utiliser pour négocier les SA de phase 2. 
La phase 2 est un échange en Quick mode. 
L'échange se fait en trois messages. 
Lors de l'échange, il est possible de négocier plusieurs SA en même temps. 
\section{Secure sockets layer - SSL}
Netscape a lancé SSL 1 en 1994, dans le but de sécuriser des transactions réalisées avec leur navigateur. 
Dans la même année, SSL 2 était déjà en route. 
Mais le protocole montrait déjà des vulnérabilités.
Fin 1995, SSL 3 était lancé. 
Il s'agissait d'une version complètement récrite de SSL, qui introduisait de nouvelles fonctionnalités issues de PCT\footnote{Microsoft's \textit{Private Communications Technology}}.
Bien que les machines actuelles intègrent SSL 3, elles essaient d'abord de négocier une connexion en SSL 2.

SSL utilise des suites de chiffrement. 
Ces suites se composent de trois fonctions de chiffrement : la méthode d'échange de clé, l'algorithme de chiffrement et une méthode de hachage. 
Il existe un large ensemble de suite, les clients envoient la liste des suites qu'ils savent gerer au serveur. 
Ce dernier choisit, dans la liste, une suite qu'il implémente. 

OpenSSL est l'implémentation la plus courante de SSL. 
Cette implémentation possède un interface en ligne de commande, qui permet de générer des clés RSA, signer des certificats, calculer des valeurs de hash, etc..

Dans un effort de standardisation de SSL, l'IETF a lancé le protocole TLS\footnote{Transport Layer Security}.
Il se base principalement sur SSL 3 bien qu'il ne soit pas compatible avec ce dernier.
L'utilisation de SSL est déconseillé suite à de nombreuses vulnérabilités détectées ces derniers temps (FREAK (2015), Hearthbleed (2014), BEAST (2013), etc.).

\subsection{Le protocole SSL}
SSL est un protocole de la couche transport, il utilise donc les protocoles de cette couche pour le transfert des données. 
Pour éviter des problèmes lors de la transmission des données, SSL utilise le protocole TCP.

De manière analogue à TCP, une session SSL se divise en trois phase (voir Fig.\ref{fig:ssl} p.\pageref{fig:ssl}) : 
\begin{enumerate}
	\item \'Etablissement de la connexion
	\item Transfert des données
	\item Clôture de la connexion.
\end{enumerate}
\begin{figure}[ht]
\centering
\begin{tikzpicture}
	\draw[-,ultra thick] (0,10.8) node [above] {Client} -- (0,0) ;
	\draw[-,ultra thick] (6,10.8) node [above] {Serveur}-- (6,0);
	\draw[arrow] (0,10) -- (6,10) node [midway,above] {Handshake : ClientHello};
	\draw[arrow] (6,9) -- (0,9) node [midway,above] {Handshake : ServerHello};
	\draw[arrow] (6,8.4) -- (0,8.4) node [midway,above] {Handshake : Certificate};
	\draw[arrow] (6,7.8) -- (0,7.8) node [midway,above] {Handshake : ServerHelloDone};
	\draw[arrow] (0,6.8) -- (6,6.8) node [midway,above] {Handshake : KeyExchange};
	\draw[arrow] (0,6.2) -- (6,6.2) node [midway,above] {ChangeCipherSpec};
	\draw[arrow] (0,5.6) -- (6,5.6) node [midway,above] {Handshake : Finished};
	\draw[arrow] (6,4.6) -- (0,4.6) node [midway,above] {ChangeCipherSpec};
	\draw[arrow] (6,4) -- (0,4) node [midway,above] {Handshake : Finished};
	\draw[|-|,ultra thick] (6.5,10.8) -- (6.5,3.8) node [midway,right] {\'Etablissement de la connexion};
	\draw[arrow] (0,2.6) -- (6,2.6);
	\draw[arrow] (6,2) -- (0,2) node [midway,above] {Application Data};
	\draw[|-|,ultra thick] (6.5,2.8) -- (6.5,1.8) node [midway,right] {Transfert des données};
	\draw[arrow] (0,0.8) -- (6,0.8);
	\draw[arrow] (6,0.2) -- (0,0.2) node [midway,above] {Alert Close Notify};
	\draw[|-|,ultra thick] (6.5,1) -- (6.5,0) node [midway,right] {Clôture de la session};
\end{tikzpicture}
\caption{Session SSL}
\label{fig:ssl}
\end{figure}
La session commence par le \textit{triple handshake}. 
Le client envoie un message ClientHello, qui indique la version de SSL supportée, la liste des suites de chiffrement et les algorithmes de compression. 
La version de SSL est signalée par deux champs dans l'en-tête : version mineur et version majeure. 
SSL3 a une version majeure de 3 et une version mineure de 0, et TLS a un version majeure de 3 et une version mineure de 1. 

Le serveur répond par trois messages.
\begin{enumerate}
	\item Le message \textit{ServerHello} indique au client la suite de chiffrement et l'algorithme de compression à utiliser.
	\item Le certificat du serveur permet au client de vérifier l'identité du serveur et contient la clé publique du serveur. Cette clé va servir à générer les différents clés pour la session.
	\item Le message \textit{ServerHelloDone} précise la fin de la séquence \textit{Hello}.
\end{enumerate}
Suite à ces trois messages, le client donne au serveur des inputs pour la génération des clés (\textit{ClientKeyExchange}), signal au serveur qu'il utilise les nouvelles clés pour le chiffrement et l'authentification (\textit{ChangeCipherSpec}) et qu'il a fini le handshake (\textit{Finished}). Le serveur répond avec son message \textit{ChangeCipherSpec} et son \textit{Finished}.
 
Le client et le serveur sont capables de s'échanger des données de façon sécurisé. 

Finalement, la session est clôturée.
\section{Transport Layer Security - TLS}
TLS est le successeur de SSL.
À l'heure actuelle, TLS est en version 1.2\footnote{RFC 5246 - \url{http://tools.ietf.org/html/rfc5246}}. 
La version 1.3 est en \textit{draft}.
De la même manière que SSL, TLS a pour objectif de sécuriser les communications entre deux entités sur un réseau.

\subsection{Détails du protocole}
TLS se base sur des enregistrements pour l'échange de données, appelé "TLS Record".
Un enregistrement est caractérisé par un champ \textit{content type}, un champ version et un champ longueur.

Avant de pouvoir échanger des données applicatives, TLS réalise le "TLS Handshake".

\subsubsection{TLS Record Protocol}
Il sert à négocier une connexion sécurisée et privée entre le client et le serveur.
Bien qu'il peut être utilisé sans chiffrement, il utiliser des clés symétriques pour s'assurer que la connexion est privée.
La sécurisation de la connexion est garantie par l'utilisation d'une fonction de hachage.

\subsubsection{TLS Handshake Protocol}
Il permet l'authentification lors du début de la communication entre le serveur et le client.
Il utilise le même handshake que SSL, mais il fournit une authentification du serveur et, en option, du client.

\subsection{Différence entre SSL et TLS}
Les deux protocoles sont semblables, mais ne sont pas compatibles.
Il existe des différences entre eux.

TLS utilise un MAC\footnote{Message Authentication Code} standardisé.
Il n'est plus limité à MD5 ou SHA, et de part sa conception, il est capable d'utiliser n'importe quelle fonction.

Ils supportent des suites de chiffrements différentes.
SSL supporte RSA, Diffie-Hellman et Fortezza.
TLS ne supporte pas Fortezza, mais il est possible d'ajouter de nouvelle suite de chiffrement au futur version du protocole.
\section{Secure Shell - SSH}
SSH a pour objectif de créer une connexion sécurisée.
De la même manière que SSL, SSH est un protocole de la couche applicative et utilise TCP.
Par contre les applications ne doivent pas forcément intégrer SSH pour être utilisées via SSH.

SSH est principalement utilisé pour remplacer \texttt{telnet}, mais il est également possible de faire du VPN.
En effet, SSH fournit de l'authentification et du chiffrement pour les communications entres les machines.

Les tunnels VPN SSH sont peu utilisés, car ils manquent de performance.
Mais ils sont simple à mettre en place.
\section{Secure Socket Tunneling Protocol - SSTP}
SSTP est utilisé pour transporter du trafic PPP/L2TP via du SSL3.
Son avantage réside dans le fait qu'il peut passer à travers les NAT, les proxys et les firewalls.

Il n'a été conçu que pour fournir des tunnels "Remote Access".
L'authentification des serveurs se fait lors de la négociation SSL.
L'authentification du client peut se faire durant cette même phase, mais elle doit se faire durant la phase PPP.
\section{HTTP over TLS/SSL - HTTPS}
\section{DirectAccess}
DirectAccess est une solution VPN sans client. 
Elle autorise un utilisateur à se connecter aux ressources de sa société.
La connexion se fait automatiquement dès que les policies sont vérifiées et que la machine dispose d'une connexion à Internet.

DirectAccess est apparu pour la première fois avec Windows 2008 R2.
Son utilisation est peu répandu, car la configuration du service en 2008 R2 était fastidieuse. 
Avec Windows 2012 R2, la configuration a été simplifiée.

Le service se base sur l'utilisation d'IPSec, mais aussi sur une implémentation de IP-HTTPS.
IP-HTTPS encapsule des paquets IPv6 dans des paquets Https. 

L'utilisation de DirectAccess impose l'utilisation d'adresse IPv6.
Mais si le réseau n'utilise pas IPv6, les tunnels sont montés en utilisant des protocoles de transition IPv6 vers IPv4, comme 6to4, Teredo ou IP-HTTPS.

\subsection{Pré-requis}
DirectAccess est un produit Microsoft et ne fonctionne qu'avec du matériel Windows.
Il y a donc des pré-requis à l'utilisation du service.
\begin{itemize}
	\item Serveur dédié pour DirectAccess
	\item Utilisation d'un Domain Controller
	\item Windows Client 7 Enterprise, 7 Ultimate, 8/8.1 Enterprise uniquement
\end{itemize}

\section{Les technologies propriétaires}
Les fabricants ont développé leur propre implémentation d'accès à distance.
\subsection{Remote Desktop Services de Microsoft}


\section{Fournisseurs d'accès}
Au début les VPN reposaient sur les réseaux des fournisseurs d'accès.
Les sociétés devaient louées des lignes auprès de ces derniers. 
Ces lignes reliaient directement les sites entre eux.
\subsection{Asynchronous Transfer Mode}
ATM est une technologie WAN se basant sur l'utilisation de cellule de même taille.
Un paquet fait 5+48 bytes de longs.
L'utilisation de ces cellules est intéressante pour les flux vidéos et audio, car les délais sont minimisés.
ATM fonctionne aussi bien en PVC qu'en SVC.

Malheureusement ATM demande une plus grande bande passante par rapport à Frame Relay. 
\subsection{Frame Relay}
Frame Relay est une technologie WAN de couche 2 sans broadcast.
Elle a besoin au strict minimum que d'une seule interface sur le routeur pour connecteur plusieurs sites distants.
Elle utilise la notion de PVC pour créer les connexions, ce qui permet le transport de la voix et des données.
Les PVCs sont différenciés par un identifiant, le DLCI.
La combinaison des deux assurent une transmission bidirectionnelle d'un ETTD à un autre.

\subsection{Multiprotocol Label Switching}
MPLS est une technologie WAN à haute-performance.
Elle permet le transfert de données en se basant sur le plus court chemin entre deux routeurs plutôt que sur les adresses IP. 
Elle est capable de transporter une grande partie des protocoles existants, comme IPv4, Ethernet, ATM, Frame Relay, ...

Elle labélise les paquets pour le routeur. 
Le label permet de trouver le plus court chemin.

MPLS est une technologie des providers. 
