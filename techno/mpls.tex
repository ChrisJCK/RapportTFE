\subsection{Multiprotocol Label Switching}
MPLS est une technologie WAN à haute-performance.
Elle permet le transfert de données en se basant sur le plus court chemin entre deux routeurs plutôt que sur les adresses IP. 
Elle est capable de transporter une grande partie des protocoles existants, comme IPv4, Ethernet, ATM, Frame Relay, ...

Elle labélise les paquets pour le routeur. 
Le label permet de trouver le plus court chemin.

Lors de son entrée dans un réseau MPLS, le routeur ajoute au paquet un label.
Ce dernier est conservé tout au long du transit dans le réseau MPLS.
Il est ajouté après l'en-tête Ethernet (couche 2) et avant l'en-tête IP (couche 3).
Le dernier routeur du réseau enlèvera le label.

Le routage se fait sur base des labels.
Ainsi, chaque routeur du réseau MPLS possède une table de commutation permettant de traiter les paquets.