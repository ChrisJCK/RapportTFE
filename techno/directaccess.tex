\section{DirectAccess}
DirectAccess est apparu pour la première fois avec Windows 2008 R2.
Son utilisation est peu répandu, car la configuration du service en 2008 R2 était fastidieuse. 
Avec Windows 2012 R2, la configuration a été simplifiée.

DirectAccess est une solution VPN sans client. 
Elle autorise un utilisateur à se connecter aux ressources de sa société.
La connexion se fait automatiquement dès que les policies sont vérifiées et que la machine dispose d'une connexion à Internet.

Le service se base sur l'utilisation d'IPSec, mais aussi sur l'implémentation de IP-HTTPS.
IP-HTTPS encapsule des paquets IPv6 dans des paquets Https. 

\subsection{Pré-requis}
DirectAccess est un produit Microsoft et ne fonctionne qu'avec du matériel Windows.
Il y a donc des pré-requis à l'utilisation du service.
\begin{itemize}
	\item Serveur dédié pour DirectAccess
	\item Utilisation d'un Domain Controller
	\item Windows Client 7 Enterprise, 7 Ultimate, 8/8.1 Enterprise uniquement
\end{itemize}
