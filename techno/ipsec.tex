\section{Internet Protocol Security - IPSec}
IPsec est un ensemble de protocole visant à sécuriser les données au niveau de la couche réseau. 
Il est composé de trois protocoles : AH (\textit{Authentication Header}), ESP (\textit{Encapsulating Security Payload}) et IKE (\textit{Internet Key Exchange}).
IKE est utilisé lors de la négociation des paramètres du tunnel VPN. 
Les deux autres protocoles fournissent la sécurité des données en les encapsulant au sein du tunnel VPN. 

\subsection{Les protocoles AH et ESP}
Le protocole AH fournit une authentification sur la source du paquet, l'intégrité des données et une protection contre les attaques par rejeu. 
Le protocole ESP fournit les mêmes sécurités que AH, sauf qu'il n'authentifie l'en-tête IP des paquets. 
Il fournit en plus la confidentialité des données en chiffrant une partie du paquet. 
AH et ESP peuvent travailler en mode tunnel ou en mode transport. 

Le mode transport est utilisé lorsque les terminaisons du tunnel VPN sont les destinataires finaux de la communication. 
Ce mode offre une sécurité de bout en bout. 
Au niveau des paquets, les en-têtes AH et ESP sont placés après l'en-tête IP, il n'y a donc qu'un seul en-tête IP.

Le mode tunnel est plus souple, mais il consomme plus de bande passante. 
Les destinataires finaux sont connectés via des passerelles VPN. 
Il n'y a pas de différence entre les VPN site-à-site et les VPN client-à-site. 
Les passerelles VPN ont pour objectif d'encapsuler les paquets pour les faire passer dans le tunnel VPN. 
Nous trouvons donc dans le paquet encapsulé l'en-tête IP du paquet d'origine (voir Fig.\ref{fig:encapIPS} p.\pageref{fig:encapIPS}). 
Comme le paquet est passé à travers la passerelle, il possède un deuxième en-tête IP qui sert au routage entre les deux passerelles.
\begin{figure}[ht]
\centering
\begin{tikzpicture}
	\matrix[matrix of nodes,]{
	\node [header] {IP};
	&
	\node [header] {Payload};
	& & & &
	\node {Paquet d'origine};\\[0.7cm]
	\node [header] {IP};
	&
	\node [header] {AH};
	&
	\node [header] {Payload}; 
	& & &
	\node {AH, mode transport};\\[0.7cm]
	\node [header] {IP};
	&
	\node [header] {ESP};
	&
	\node [header] {Payload (chiffré)}; 
	&
	\node [header] {ESP};
	&	&
	\node {ESP, mode transport};\\[0.7cm]
	\node [header] {IP (tunnel)};
	&
	\node [header] {AH};
	&
	\node [header] {IP}; 
	&
	\node [header] {Payload};
	&&
	\node {AH, mode tunnel};\\[0.7cm]
	\node [header] {IP (tunnel)};
	&
	\node [header] {ESP};
	&
	\node [header] {IP (chiffré)}; 
	&
	\node [header] {Payload (chiffré)};
	&
	\node [header] {ESP};
	&
	\node {ESP, mode tunnel};\\
};
\end{tikzpicture}
\caption{Schéma des encapsulations IPSec}
\label{fig:encapIPS}
\end{figure} 

\subsection{Security association}
Quand nous parlons de monter un tunnel, en réalité, nous synchronisons un état partagé entre les terminaisons du tunnel. 
Cet état partagé se nomme une SA (\textit{security association}) en IPSec. 
Une SA contient l'algorithme de chiffrement utilisé et les clés utilisées, l'algorithme d'authentification, un numéro d'identifiant, le \textit{security parameter index} (SPI), … 
Plus d'autres paramètres qui servent à maintenir les tunnels VPN. 
Les SA peuvent être créées manuellement ou gérées par l'IKE. 
Chaque terminaison possède deux SA, une pour le trafic entrant et une pour le trafic sortant. 
De plus, chaque pair est lié à un protocole. 
Les SA se caractérisent par un triplet formé du SPI, de l'adresse de destination et du protocole. 
Les SA sont stockées dans une SAD (\textit{security association database}). 
Cette SAD est utilisée pour déterminer quel protocole est utilisé pour les paquets sortants et pour fournir les paramètres pour déchiffrer et/ou authentifier les paquets entrants. 
Il est possible de combiner les SA pour créer des tunnels VPN complexe. 

Les SA sont des éléments simples, c'est-à-dire qu'elles traitent tous les paquets de la même manière. 
Pour un réglage plus fin, IPSec utilise des policies. 
Ces policies se basent sur les champs suivants des en-têtes du paquet.
\begin{itemize}
\item L'adresse de destination
\item L'adresse source
\item Le protocole de la couche transport
\item Le port source
\item Le port de destination
\end{itemize}
Elles servent à déterminer quels paquets à émettre sur quel tunnel, à dropper les paquets ne correspondant à aucune des règles décrites dans les policies. 
De la même manière que les SA, les policies sont stockées dans une SPD (\textit{security policy database}). 
Le fonctionnement est similaire, pour chaque paquet entrant ou sortant, le système consulte la SPD pour déterminer les règles à appliquer au paquet. Si une règle est trouvé, le système cherche après la SA correspondante.

\subsection{Le protocole IKE}
IKE a un seul objectif : procéder à des échanges de clé Diffie-Hellman pour sécuriser un tunnel VPN. 
Il négocie le chiffrement, l'authentification nécessaire au tunnel, qui satisfont les policies. 

IKE dérive du \textit{Internet Security Association and Key Management} Protocol (ISAKMP). 
ISAKMP est un framework qui fournit des outils pour la sécurisation des échanges et l'échange de clé. 
De plus, IKE utilise différents mode du protocole OAKLEY.
Il établit une SA en deux phases et il possède cinq modes d'échange, dont trois découlent ISAKMP.
Les deux derniers modes ne sont utilisés que lors de la phase deux.

\subsubsection{La phase 1 d'IKE}
La phase 1 crée un canal sécurisé entre les terminaux du tunnel pour déterminer les SA. Le canal sécurisé est créé après l'authentification des terminaux. Les SA de la phase 1 sont bidirectionnelles, c'est-à-dire qu'une SA sécurise le trafic entrant et sortant. 
Pour l'échange des SA de la phase 1, IKE possède deux modes d'échanges : 
\begin{itemize}
	\item Main mode
	\item Agressive mode
\end{itemize}
Pour l'authentification d'un terminal, il existe quatre méthodes : 
\begin{itemize}
	\item a shared secret 
	\item a digital signature
	\item public key encryption
	\item revised public key encryption
\end{itemize}
Le mode \textit{"main"} d'IKE travaille en trois étapes (voir Fig.\ref{fig:ipsmain} p.\pageref{fig:ipsmain}). 
\begin{figure}[ht]
\centering
\begin{tikzpicture}
	\draw[-,ultra thick] (0,7) node [above] {Initiator} -- (0,0) ;
	\draw[-,ultra thick] (7,7) node [above] {Responder}-- (7,0);
	\draw[arrow] (0,6) -- (7,6) node [midway,above] {HDR - SA};
	\draw[arrow] (7,5) -- (0,5) node [midway,above] {HDR - SA};
	\draw[arrow] (0,4) -- (7,4) node [midway,above] {HDR - KE - NONCE$_i$};
	\draw[arrow] (7,3) -- (0,3) node [midway,above] {HDR - KE - NONCE$_r$};
	\draw[arrow] (0,2) -- (7,2) node [midway,above] {HDR - ID$_i$ - \textit{AUTH}};
	\draw[arrow] (7,1) -- (0,1) node [midway,above] {HDR - ID$_r$ - \textit{AUTH}};
\end{tikzpicture}
\caption{IPsec : mode "main"}
\label{fig:ipsmain}
\end{figure}
Premièrement, l'initiateur envoie un message contenant une liste des méthodes de sécurisation qu'il utilise. 
Le receveur choisit dans la liste reçue la méthode correspondant à ses policies et envoie sa décision à l'initiateur. 
Ensuite, ce dernier envoie sa clé privée pour créer le secret partagé de l'algorithme de Diffie-Hellman. 
Le receveur fait de même. Ils sont donc capables tous les deux de créer les clés. 
Les clés dépendent des méthodes d'authentification choisies. 
Finalement, l'initiateur envoie son identité et des informations sur l'authentification. 
Ces messages sont chiffrés et masquent donc l'identité des terminaux. 
L'échange se fait en six messages.

À la fin de ce mode, les terminaux sont d'accord sur les algorithmes de chiffrement et de confidentialité des données. 
Ils possèdent également les clés pour les algorithmes sélectionnés. 

Le mode \textit{"agressive"} fait le même travail de façon plus rapide, il n'utilise que trois messages (voir Fig.\ref{fig:ipsagg} p.\pageref{fig:ipsagg}).
\begin{figure}[ht]
\centering
\begin{tikzpicture}
	\draw[-,ultra thick] (0,4) node [above] {Initiator} -- (0,0) ;
	\draw[-,ultra thick] (8,4) node [above] {Responder}-- (8,0);
	\draw[arrow] (0,3) -- (8,3) node [midway,above] {HDR - SA - KE - NONCE$_i$ - ID$_i$};
	\draw[arrow] (8,2) -- (0,2) node [midway,above] {HDR - SA - KE - NONCE$_r$ - ID$_r$ - \textit{AUTH}};
	\draw[arrow] (0,1) -- (8,1) node [midway,above] {HDR - \textit{AUTH}};
\end{tikzpicture}
\caption{IPsec : mode "agressive"}
\label{fig:ipsagg}
\end{figure} 
Lors du premier envoi, l'initiateur émet la liste des méthodes de sécurisation, son identité et sa clé. 
Le receveur répond par son choix de sécurisation, sa clé, son identité et ses identifiants. 
Finalement l'initiateur s'authentifie auprès du receveur. 
Ce dernier message peut être chiffré.
\subsubsection{La phase 2 d'IKE}
Une fois la SA établie, les terminaux peuvent l'utiliser pour négocier les SA de phase 2. 
La phase 2 est un échange en Quick mode. 
L'échange se fait en trois messages. 
Lors de l'échange, il est possible de négocier plusieurs SA en même temps. 