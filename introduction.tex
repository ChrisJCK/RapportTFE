\chapter*{Introduction}
\addcontentsline{toc}{chapter}{Introduction}
Le sujet de ce TFE m'a été proposé par mon maître de stage.
Après plusieurs recherches de mon côté et en gardant en tête son idée, j'ai trouvé le sujet intéressant.

Un VPN est un réseau privé construit sur une infrastructure publique. 
Il permet à une entreprise de connecter de manière sécurisée des sites et/ou des utilisateurs distants.
Ce n'est pas la seule utilité des VPN. 
Mais pour ce travail, je ne m'intéresse qu'à l'utilisation des VPN en entreprise.
Auparavant, les tunnels VPN étaient construits sur des réseaux privés (ATM, FrameRelay, MPLS) que les entreprises louaient auprès des fournisseurs de télécommunication.
De nos jours, l'infrastructure publique utilisée est le réseau Internet pour des raisons financières et de disponibilité.

Les données transitant par l'Internet peuvent être interceptées.
En cas d'interception, les données sont lisibles par le pirate, à moins que les données soient chiffrées avec un protocole de chiffrement robuste.
La sécurisation des VPN est un enjeu essentiel pour éviter la fuite d'information sensible.

L'utilisation de VPN sécurisé est un élément important des entreprises pour la productivité des utilisateurs distants. 
Un utilisateur distant ne se résume plus à un commercial visitant des sièges en divers lieux.
On y trouve tous les types d'employées. Par exemple, le télé-travailleur qui veut échapper aux embouteillages sur la route; le consultant informatique qui doit faire une intervention chez un client; tout employé souhaitant lire ses mails sur son smartphone;...

Les philosophies de sécurisation étant différentes d'une solution à une autre, un comparatif des solutions professionnelles d'accès distant va apporter une vision plus claire sur les fonctionnalités.
Ce comparatif va mettre en évidence les différences et les ressemblance entre les solutions des leaders commerciaux dans le domaine.

L'objectif est d'établir un comparatif complet et précis, mettant en évidence les différences et les similarités entre les différents solutions.
Le comparatif se fait sur plusieurs critères pertinents. 
Ces critères sont explicités plus loin de le rapport.\\

Ce rapport est composé de trois grandes parties.

La première partie se découpe en trois chapitres.
Le premier chapitre décrit les besoins des entreprises en terme d'accès distant.
Le chapitre suivant présente la situation actuelle des entreprises.
Le dernier chapitre de cette partie contient la méthodologie utilisée pour réaliser ce travail de fin d'étude.

La seconde partie se penche sur les technologies utilisées dans les accès distants.
On commence par une description d'une partie des technologies disponibles.
On finit par une brève comparaison de ces technologies sur différents critères.

La troisième partie est consacrée à la comparaison des solutions commerciales.
Le premier chapitre décrit le choix des solutions.
Le deuxième chapitre porte sur la configuration des solutions, dont l'infrastructure et la passerelle.
Le troisième chapitre explique le scénario suivi pour les tests.
Le quatrième chapitre décrit les critères sélectionnés et présente le tableau de comparaison ainsi que les commentaires.

Finalement, la conclusion présente des recommandations sur base de mes commentaires et de mon ressenti sur les différents solutions testées.
