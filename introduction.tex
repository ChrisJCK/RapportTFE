\chapter*{Introduction}
\addcontentsline{toc}{chapter}{Introduction}
Un VPN est un réseau privé construit sur une infrastructure publique. 
Il permet à une entreprise de connecter de façon sécurisée des sites et/ou des utilisateurs distants.
Ce n'est pas la seule utilité des VPN. 
Mais pour ce travail, je ne m'intéresse qu'à l'utilisation des VPN en entreprise.
Auparavant, les tunnels VPN étaient construits sur des réseaux privés que les entreprises louaient auprès des fournisseurs de télécommunication.
De nos jours, l'infrastructure publique utilisée est le réseau Internet pour des raisons financières et de disponibilité.

Les données transitant par l'Internet peuvent être interceptées.
En cas d'interception, les données sont lisibles par le pirate, à moins que les données soient chiffrées avec un protocole de chiffrement robuste.
La sécurisation des VPN est un enjeu essentiel pour éviter la fuite d'information sensible.

L'utilisation de VPN sécurisé est un élément important des entreprises pour la productivité des utilisateurs distants. 
Un utilisateur distant ne se résume plus à un commercial visitant des sièges en divers lieux.
On y trouve tous les types d'employées. Par exemple, le télé-travailleur qui veut échapper aux embouteillages sur la route; le consultant informatique qui doit faire une intervention chez un client.

Les philosophies de sécurisation étant différentes d'une solution à une autre, un comparatif des solutions professionnelles ... 