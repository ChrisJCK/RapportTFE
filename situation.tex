\part{La problématique des accès distants en entreprise}
\chapter{Description des besoins des entreprises}
\section{Les besoins habituels}
Les entreprises possèdent toute au moins un serveur contenant les fichiers de travail, ainsi qu'un serveur mail.
Ces deux ressources sont primordiales pour l'entreprise et pour les employés.
L'accessibilité et la disponibilité de ces deux éléments sont capitales.

L'accès aux mails depuis l'extérieur du réseau de l'entreprise est un avantage pour les employés.
Ainsi un employé, se trouvant en voyage d'affaire ou en déplacement chez un client, est à même de rester au courant des dernières nouvelles.
Il peut aussi correspondre avec ses collègues, transmettre des informations,...
Mais une mauvaise utilisation des mails peut mener à des problèmes sérieux.
L'utilisation des mails professionnels doit servir uniquement à communiquer avec les collègues, les clients, les fournisseurs.

L'accès aux fichiers est tout aussi important, mais l'accessibilité aux fichiers depuis l'extérieur doit être contrôlé.
En effet, les fichiers critiques (bancaire, technologique, brevet, confidentiel,...) ne devraient pas être disponibles depuis l'extérieur.
Ces fichiers sensibles pourraient mettre à mal la réputation, le fonctionnement de l'entreprise.

La mise à disposition de ces ressources aux employés depuis l'extérieur doit se faire en suivant des règles respectant la principe du AAA.
Les utilisateurs doivent s'authentifier pour obtenir l'accès aux ressources.
Les opérations effectuées par les utilisateurs sont sauvegardées à des fins de résolution de problèmes.
\section{Les besoins spécifiques}
En plus des fichiers et des mails, certaines entreprises peuvent mettre à disposition des employés un intranet et certaines applications.

L'intranet est, en général, un site web, accessible uniquement en interne, qui permet soit de fournir des nouvelles, soit de collaborer avec ses collègues,...
Les informations qui s'y trouvent peuvent être soit d'ordre public, comme l'interview de Monsieur X dans le magazine Y, soit d'ordre privé, comme une discussion entre collègues sur un problème rencontré.

Des applications métiers (SAP, CRM,...) peuvent être aussi accessible depuis l'extérieur.

De la même manière que pour les mails et les fichiers, l'accès doit être contrôlé.

Le contrôle des accès est la première ligne de défense de la sécurité.
Par la suite, il est possible d'ajouter d'autres lignes de défense, comme un deuxième processus d'authentification, le chiffrement des données.

\chapter{La situation actuelle}
La plupart des entreprises actuelles sont confrontées à la problématique des accès distants.
En effet, un grande nombre d'employé ont la possibilité de travailler de chez eux ou d'un autre lieux hors de l'entreprise à partir du moment où il dispose d'une connexion à Internet.
De plus, les employés possèdent généralement des ordinateurs portables pour leur travail.
Il est utile qu'ils aient alors accès à leur document de travail à tout moment via cet ordinateur.
Si ce n'est pas le cas, ils se demanderaient l'utilité de posséder un portable.

On estime qu'en Belgique 20\% des travailleurs utilisent le télétravail.
Ce chiffre est en augmentation, mais les entreprises sont réticentes même si les mentalités changent.
Le télétravail offre des avantages, comme le gain de temps dans les déplacements, une meilleure gestion du temps,...
Mais pour l'entreprise, le télétravail comporte des risques.
Un employeur ne sait pas forcément localiser ses employés, mais surtout il ne sait pas vérifier l'utilisation des données.
Il est donc normal que l'employeur mette des restrictions quant à l'accès aux ressources.
\chapter{Méthodologie}
Pour réaliser ce TFE, j'ai commencé par m'approprier les connaissances théoriques nécessaire à la création d'un tunnel VPN.
C'est-à-dire que j'ai étudié l'ensemble des protocoles utilisés, en me focalisant sur les principaux à savoir IPSec, SSL et TLS.

Cette première analyse théorique m'a permis de cerner les difficultés à créer un tunnel sécurisé.

En plus des technologies, j'ai lu les guides d'administration des principales solutions que j'allais utiliser.
Mais avant de les configurer pour créer les tunnels VPN, je les ai testé pour m'assurer du fonctionnement.

En me basant sur l'infrastructure à ma disposition et les bonnes pratiques en architecture des réseaux, j'ai simulé un réseau d'entreprise contenant les composants habituels qui sont mis en œuvre.
Avant de commencer les tests, une vérification des fonctionnalités de l'infrastructure a été nécessaire.
J'ai vérifié, en interne, que mes ordinateurs clients avaient accès via Outlook au serveur Exchange,  que les GPOs s'appliquaient aux groupes concernés, que les accès sur les fichiers du DFS étaient contrôlés.
En bref, je me suis assuré que mon réseau fonctionnait de manière cohérente et était disponible pour un utilisateur.

Pour la configuration des passerelle et des équipements réseaux, je me suis basé sur les guides d'administration et les procédures internes de mon lieu de stage.

Ensuite, j'ai réalisé les tests en simulant l'accès distant depuis une machine se trouvant dans le réseau du laboratoire.
Comme l'architecture variait entre les différentes solutions, j'ai testé les solutions l'une après l'autre.
Lors des tests, je prenais note des résultats sur un tableau.
Ce tableau a servi de base au tableau de comparaison final.

Une fois les tests terminés, j'ai réalisé le comparatif sur des bases des critères sélectionnés et des notes prises lors des tests. 