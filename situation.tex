\part{La problématique des accès distants en entreprise}
\chapter{Les besoins habituels}
\chapter{Les besoins spécifiques}
\chapter{La situation actuelle}
\chapter{Méthodologie et objectifs}
Pour réaliser ce TFE, j'ai commencé par m'approprier les connaissances théoriques nécessaire à la création d'un tunnel VPN.
C'est-à-dire que j'ai étudié l'ensemble des protocoles utilisés, en me focalisant sur les principaux à savoir IPSec, SSL et TLS.

Cette première analyse théorique m'a permis de cerner les difficultés à créer un tunnel sécurisé.

En plus des technologies, j'ai lu les guides d'administration des principales solutions que j'allais utiliser.
Mais avant de les configurer pour créer les tunnels VPN, je les ai testé pour m'assurer du fonctionnement.

En me basant sur l'infrastructure à ma disposition et les bonnes pratiques en architecture des réseaux, j'ai simulé un réseau d'entreprise contenant les composants habituels qui sont mis en œuvre.

Ensuite, j'ai réalisé les tests.
Comme l'architecture variait entre les différentes solutions, j'ai testé les solutions l'une après l'autre. 

Une fois les tests terminés, j'ai réalisé le comparatif sur des bases des critères sélectionnés. 