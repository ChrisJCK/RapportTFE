\section{Scénario de test}
Le but de ce travail est de tester les accès distants aux ressources d'une entreprise.
Dans les points précédents, j'ai décrit l'ensemble des ressources et des méthodes d'accès.

Dans ce chapitre, j'explique le scénario de test pour comparer les solutions d'accès sur des points similaires.

La première ressource à contacter est le serveur mail de l'entreprise. 
Pour ce faire, Outlook est installé sur la machine distante. 
Il est configuré pour se connecter à la boîte mail de \textit{c.juckler@labotfe.be}.
Cette boîte mail n'est accessible que depuis le réseau de l'entreprise.

Les fichiers sont la deuxième ressource à accéder.
Un disque mappé est configuré par GPO aux utilisateurs du domaine. 
Ce disque est un lien vers le \textit{DFS Namespace} de l'entreprise, à savoir \texttt{\textbackslash\textbackslash labotfe.be \textbackslash Data}

La troisième ressource est l'intranet de l'entreprise.
Cet intranet est composé d'une seule page web, qui affiche un message de bienvenue.

Le dernier test porte sur l'utilisation d'\textit{ActiveSync}.
ActiveSync permet de synchroniser la boîte mail d'un utilisateur sur son smartphone.
Pour ce test, on m'a prêté un smartphone Nokia Lumia 520 avec Windows Phone 8.

