Les critères de comparaison vont permettre de classer les différentes solutions.
Les critères sélectionnés sont de plusieurs types.
Il y a des critères liés aux tests de connectivité, d'autres sont liés aux fonctionnalités ou aux technologies utilisées.

\section{Les critères sélectionnés}
Il existe un nombre importants de critères pour réaliser ce comparatif.
J'ai décidé d'en choisir seize qui montrent bien les différences entre les différentes solutions sélectionnées. 
Mes choix se sont portés sur d'une part des tests de connectivité, vu que l'un des objectifs essentiels des VPNs est l'accès aux ressources.
D'autre part, ils portent sur les suites de chiffrement et les protocoles supportés.
Un autre objectif des VPN est la sécurisation des données. 
De plus, un critère plus personnel sur l'utilisation des systèmes est intéressante.
Car n'ayant aucune expérience avec ce genre de systèmes, j'ai dû apprendre à les utiliser.
Finalement, un critère financier est important à signaler.
\rowcolors{2}{lightgray}{}
\begin{longtable}{p{4cm}|p{12cm}}
	\toprule
	\textbf{Critère} & \textbf{Description} \endhead
    \hline 
    Connexion Outlook & Test de connectivité entre le client mail Outlook et le serveur Exchange \\
    \hline
    Connexion Outlook Web App & Test de connectivité entre la machine cliente et la Web App du serveur Exchange. Dans ce cas, l'OWA n'est accessible que depuis le réseau interne, il faut donc que les machines distantes aient un accès au réseau interne.\\
    \hline
    Connexion au DFS Namespace & Test de connectivité entre la machine cliente et le DFS Namespace \\
    \hline
    Connexion à l'intranet & Test de connectivité entre la machine cliente et le serveur Web interne, qui héberge l'intranet \\
    \hline
    Type de tunnel & La connexion se fait par un tunnel VPN, mais quel type de tunnel ?. Les réponses possibles sont SSL ou ESP/IPSec.\\
    \hline
    Algorithme de chiffrement par défaut & Lors de la configuration, un algorithme ou un ensemble d'algorithme est sélectionné par défaut.\\
    \hline
    Version de SSL/TLS par défaut & Il existe plusieurs versions de ces protocoles. Le but est de savoir le(s)quelle(s) a/ont été sélectionné par défaut.\\
    \hline
    Algorithme de chiffrement offrant le plus de sécurité & Lors de la configuration, il est possible de sélectionner le(s) algorithme(s) souhaité(s). Ce critère met en lumière l'algorithme le plus sécurisé supporté par le système.\\
    \hline
    Version de SSL/TLS maximum supportée & Le but est de savoir qu'elle est la version maximale de la technologie qui est supportée par le système.\\
    \hline
    Facilité de configuration & Ce critère est plus personnel, mais il se base sur un code que j'explique plus tard.\\
    \hline
    Fallback SSL & Ce test vérifie la présence d'une option pour le passage de IPSec vers SSL en cas de problème lors de la connexion en IPSec.\\
    \hline
    Host Checker & Ce test signale la présence d'une méthode pour vérifier que l'hôte peut se connecter en VPN.\\
    \hline
    Méthode d'authentification & Ce test précise le mode d'authentification utilisé lors des tests, ainsi que les autres méthodes d'authentification supportées.\\
    \hline
    Split Tunneling & Signale la présence ou non de l'option. Si oui, le fonctionnement par défaut.\\
    \hline
    Push Mail & Ce test porte sur l'utilisation d'ActiveSync pour le push mail.\\
    \hline
    Prix & Le prix tient compte du prix des appareils nécessaires à la réalisation de l'infrastructure.\\
    \bottomrule
    \caption{Critères de comparaison}
	\label{tab:criteres}\\
\end{longtable}

\subsection{Explication du critère "Facilité de configuration"}
Ce critère est plus personnel, mais il se base sur les points suivants.
J'attribue une note entre 1 et 5 sur la configuration et une note entre 1 et 5 sur l'interface graphique.

Pour la configuration, la note de 5 signifie qu'elle est simple.
C'est-à-dire que le guide d'administration suffit largement pour configurer les différents éléments à mettre en place.
Par contre, une note de 1 signifie que la configuration n'est pas simple.
Il faut chercher plus loin que le guide d'administration fourni, voir demander sur des forums pour trouver la bonne marche à suivre.

Pour l'interface graphique, une note de 5 signifie que l'interface est intuitive. 
Il est relativement aisé de trouver les éléments à modifier, les menus sont bien présentés.
À l'inverse, une note de 1 montre que l'interface est mal conçue.

\section{Grille de comparaison}
Sur base des critères et des différentes architectures/solutions, j'ai dressé un tableau de comparaison de ces différentes solutions.

\begin{tabular}{m{4cm}m{12cm}}
\ok & La connexion est établie / présence de l'option \\
\nok & La connexion n'est pas établie / option non disponible \\
\unk & L'information n'est pas disponible \\
N/A & Le test ne s'applique pas à cette solution \\
\end{tabular}

\begin{landscape}
\begin{longtable}{>{\centering\columncolor{lightgray}}m{4cm}|>{\centering}m{3.3cm}|>{\centering}m{3.3cm}|>{\centering}m{3.3cm}|>{\centering}m{3.3cm}|m{3.3cm}<{\centering}}
	\toprule
	\textbf{Critère} & \textbf{Juniper - Portail Web} & \textbf{Juniper - Pulse Secure} & \textbf{Cisco Clientless} & \textbf{Cisco AnyConnect} & \textbf{DirectAccess} \endhead
    \hline
    Connexion Outlook & N/A & \ok & N/A & \ok & \ok \tabularnewline
    \hline
    Connexion Outlook Web App & \ok & \ok & \ok & \ok & \ok \tabularnewline
    \hline
    Connexion au DFS Namespace & \ok & \ok & \nok & \ok & \ok \tabularnewline
    \hline
    Connexion à l'intranet & \ok & \ok & \ok & \ok & \ok \tabularnewline
    \hline
    Type de tunnel & SSL & ESP & SSL & DTLS & IPSec \tabularnewline
    \hline
    Algorithme de chiffrement par défaut & 128bit et + & AES128/SHA1 & AES256-128-RC4-3DES/SHA1 & RSA-AES128/SHA1 & AES-CBC-128/SHA-256 \tabularnewline
    \hline
    Version de SSL/TLS par défaut & SSLv3 \& TLS & SSLv3 \& TLS & SSLv2, SSLv3 et TLSv1 & SSLv2, SSLv3 et TLSv1 & \unk \tabularnewline
    \hline
    Algorithme de chiffrement offrant le plus de sécurité & 168bit et + & AES256/SHA1 & AES256/SHA1 & AES256/SHA1 & TLS-ECDHE-RSA-AES256-CBC-SHA384 \tabularnewline
    \hline
    Version de SSL/TLS maximum supportée & TLSv1 & TLSv1 & TLSv1 & TLSv1 & \unk \tabularnewline
    \hline
    Facilité de configuration & 10 & 10 & 6 & 2 & 8 \tabularnewline
    \hline
    Fallback SSL & N/A & \ok (option) & N/A & \nok & N/A \tabularnewline
    \hline
    Host Checker & \ok & \ok & \ok & \ok & \ok \tabularnewline
    \hline
    Méthode d'authentification & Active Directory & Active Directory & Active Directory & Active Directory & Kerberos \tabularnewline
    \hline
    Split Tunneling & N/A & \ok (désactivé) & N/A & \ok (activé) & \ok (activé) \tabularnewline
    \hline
    Push Mail & \multicolumn{2}{c|}{\ok} & \unk & \unk & N/A \tabularnewline
    \hline
    Prix (en euro) & \multicolumn{2}{c|}{5400} & \multicolumn{2}{c|}{500-1000} & 4200 \tabularnewline
    \bottomrule
    \caption{Tableau de comparaison}
	\label{tab:comparaison}\\
\end{longtable}
\end{landscape}
\subsection{Commentaires}
\subsubsection{Solution Juniper}
La solution Juniper est la meilleure des solutions, vu qu'elle offre toute les fonctionnalités demandées, ainsi qu'une interface graphique intuitive. 
Le guide d'administration est suffisamment complet que pour réaliser la configuration de manière autonome.

La seule remarque à faire porte sur l'utilisation d'ActiveSync pour le push mail.
Lors des tests, j'employais un utilisateur, qui est administrateur du domaine.
Par défaut, le serveur Exchange n'autorise pas les administrateurs à utiliser ActiveSync.
Pour éviter de modifier les paramètres de cet utilisateur, j'ai pris un utilisateur standard pour mes tests.

\subsubsection{Solution Windows}
La solution Windows DirectAccess a son intérêt, mais elle est plus limitée.
Elle ne s'appuie que sur l'utilisation de système d'exploitation Windows.
Elle n'est pas compatible avec les appareils de type smartphone, ce qui réduit un peu l'utilité de ce service pour accéder au mail depuis l'extérieur.
Le critère "Host Checker" est sous-entendu valide, car DirectAccess est déployé via une GPO liée à un groupe de sécurité.
Donc, les machines utilisant DirectAccess sont vérifiées par l'administrateur réseaux, car il a dû placer les ordinateurs concernés dans le groupe de sécurité.

Il n'est pas possible de déterminer avec précisions quels sont les algorithmes de chiffrement utilisés par DirectAccess.
Une liste des algorithmes supportés est disponible sur l'Internet\footnote{\url{http://directaccess.richardhicks.com/2014/09/23/directaccess-ip-https-ssl-and-tls-insecure-cipher-suites/}}.
Sur cette liste, on remarque que des algorithmes sans chiffrement sont utilisables.
Néanmoins, ces algorithmes sont utilisés par IP-HTTPS.
Les paquets entre le client et le serveur sont chiffrés par IPSec, il n'est donc pas utile de chiffrer une deuxième fois les paquets.

\subsubsection{Solution Cisco}
La solution Cisco est sans doute la moins agréable à utiliser.
Les problèmes de version du logiciel sont pour le moins dérangeantes.
Pour des raisons techniques, j'ai dû travailler avec la version 8.3 du logiciel.
À la base, la version 9.3 était prévue.
Mais l'ASA ne disposant que de \numprint[MB]{256}, il m'était impossible de désarchiver le package AnyConnect nécessaire au déploiement du client sur les machines.
En cherchant sur le site de Cisco, la dernière version compatible avec l'ASA mis à disposition est la version 8.3.
Malheureusement, cette version ne gère pas bien les certificats.
J'ai donc désactivé l'authentification de l'ASA via le certificat.

De plus, en passant en version 8.3, la connexion au serveur de fichier n'est plus possible.
Avec la version 9.3, il était possible d'accèder au DFS Namespace.

Par ailleurs, la solution Cisco impose l'utilisation de Java.
Car l'ASDM et les services accessibles via le portail web se base sur Java pour fonctionner.

\subsection{Les prix des solutions}
Les trois solutions décrites ont des prix largement différents, car les techniques de ventes varient d'un fabricant à l'autre.

La solution Juniper est la plus chère avec un prix minimum de \euro{5400}.
Ce prix comprend la passerelle, plus une licence pour dix utilisateurs et le support de trois ans.

La solution Cisco est la moins chère avec un prix entre \euro{500} et \euro{1000}.
Le prix communiqué comprend le firewall avec la licence de base.

La solution Windows est entre les deux.
Son prix est plus variable selon le serveur choisi, mais la licence Windows Server coûte à elle seul \euro{1200}.
Un serveur de puissance correcte coûte dans les \euro{3000}.


